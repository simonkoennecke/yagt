%Dokumententyp
\documentclass[a4paper]{article}


\usepackage[a4paper,left=2cm, right=3cm, top=2cm]{geometry}

%Kodierung
\usepackage[utf8]{inputenc}
\usepackage[T1]{fontenc}

%Grafiken einbinden
\usepackage{graphicx}

%Position von Grafiken und Tabellen erzwingen:
\usepackage{float}

%URLs im Literaturverzeichnis
\usepackage{url}

\usepackage{amsmath}

%Schriftart Arial:
% \usepackage{helvet}

%Figures with text around it:
\usepackage{wrapfig}

\usepackage{listings}

%seitennummern rechts:
% \usepackage{fancyhdr}
% \fancyhf{} % clear all header and footers
% \renewcommand{\headrulewidth}{0pt} % remove the header rule
% \rfoot{\thepage}
% \fancypagestyle{plain}{%redefining plain pagestyle
% \fancyhf %clear all headers and footers fields
% \fancyhead[R]{\thepage} %prints the page number on the right side of the header
% }

%Schriftart Times New Roman "like"
\usepackage{txfonts}

%Sprache
\usepackage[german]{babel}

%Checkmarks: (usage: \checkmark)
\usepackage{dingbat}

\usepackage{listings}
\usepackage{color}
\definecolor{javared}{rgb}{0.6,0,0} % for strings
\definecolor{javagreen}{rgb}{0.25,0.5,0.35} % comments
\definecolor{javapurple}{rgb}{0.5,0,0.35} % keywords
\definecolor{javadocblue}{rgb}{0.25,0.35,0.75} % javadoc
 
\lstset{language=Java,
basicstyle=\ttfamily,
keywordstyle=\color{javapurple}\bfseries,
stringstyle=\color{javared},
commentstyle=\color{javagreen},
morecomment=[s][\color{javadocblue}]{/**}{*/},
numbers=left,
numberstyle=\tiny\color{black},
stepnumber=1,
numbersep=5pt,
tabsize=4,
showspaces=false,
lineskip={-1.5pt},
showstringspaces=false}

%Tabellenextras
\usepackage{tabularx}

%Zeilenabstand 1.5
\linespread{1.5}
\usepackage{setspace}

%Figure Captions mit Fußnoten
\usepackage{footnote}
%\setlength{\parindent}{0pt} 


%itemize items richtig ausrichten (nicht links überlappen!)
% \setlist{leftmargin=0}

% %%%%TITELSEITE%%%%%%(
% \title{ Konzept und Implementierung\\ eines Systems zur \\Anforderung und Verwaltung von virtuellen privaten Clustern}
% \author{\textbf{\large Bachelorarbeit}}
% 
% \date{zur Erlangung des akademischen Grades Bachelor of Science an der Universität Paderborn im Fachbereich Informatik im Studiengang Bachelor Informatik}

% %%%%TITELSEITE%%%%%%)

% \pagestyle{fancy}
\begin{document}

\title{Algorithmische Geometrie - Sommersemester 2015\\
       3. Aufgabenblatt }
\author{Simon Koennecke und Felix Bröker}
\date{}
\maketitle

\section*{Aufgabe 1 - Bewegungsplanung in der Ebene}

Gegeben sind die Hindernisse $H = \left\{p_1, \dots, p_n\right\}$, der Start $s$, das Ziel $t$ und der Radius $r$ des Roboters. Des Weiteren sind der Start- und Endpunkt im Radius des Roboters frei von Hindernissen. Die Voronoi Kanten (VK) seien die einzig möglichen Wege für den Roboter. D.h. ein großzügiges Umfahren ist mit dem Algorithmus nicht möglich.

\begin{itemize}

\item Berechne $VD(H \cup \left\{s,t\right\})$ ohne die VK ins Unendliche.

\item Prüfe für alle VK $e$ und die Punkte der angrenzenden Voronoi-Regionen $p_i$ und $p_j$, ob gilt: $d(p_i, p_j) \geq r$. In diesem Fall ist sichergestellt, dass der Roboter die Kante passieren kann, ohne mit einem der Hindernisse $p_i$ oder $p_j$ zu kollidieren. Sollte jedoch $d(p_i, p_j) < r$ gelten, so entferne $e$ aus $VD$. 

Nach Voraussetzung brauchen
die Kanten der Region $s$ und $t$ hier nicht berücksichtigt werden. 

\item Erstelle einen Graphen $G = (V,E)$, wobei $E$
der Menge der übrig gebliebenen Voronoi-Kanten entspricht.
Das Kantengewicht einer jeden Kante $e$ des Graphen $G$
wird auf den Wert $1$ festgelegt. Die Knotenmenge $V$ wird so konstruiert, dass der Startpunkt $q_s$ alle Voronoi-Kanten der Voronoi-Region $s$ vereint. Dasselbe Verfahren wird analog für den Zielpunkt $q_t$ angewandt. Alle anderen Voronoi-Konten VK werden direkt in $V$ übernommen. 

Der Graph $G$ kann zyklisch und nicht zusammenhängend sein.

\item Suche in dem Graphen $G$ den Weg $w$ zwischen $q_s$ und $q_t$. Hier kann ein Shortest Path Algorithmus angewendet werden, z.B. der Algorithmus von Dijkstra.

\item Geben den Weg $w$ aus. Sollte $w$ kein Weg sein, existiert kein kollisionsfreier Pfad mit den gegebenen Hindernissen von $s$ nach $t$.

\end{itemize}  

\section*{Aufgabe 2 - Geometrische Graphen}

\subsection*{(a) Beweisen Sie Eulers Formel}

https://www.ics.uci.edu/~eppstein/junkyard/euler/

https://de.wikipedia.org/wiki/Eulerscher\_Polyedersatz

Eulers Formel:

\textit{Sei $G$ ein planarer, zusammenhängender Graph, dann gilt:}
 $$\underbrace{|V|}_{Knoten} = \underbrace{|E|}_{Kanten} - \underbrace{|F|}_{Facetten} + 2 $$

Beweis durch strukturelle Induktion:

\begin{itemize}
  \item I.A. ( Graph mit $|V| = 1$ ): 1 = 0 - 1 + 2. \checkmark
  
	(Wenn der planare Graph nur aus einem Knoten besteht, gibt es keine Kanten und genau
	eine Außenfacette.)
  
  \item I.V.: $|V| = |E| - |F| + 2 $ gelte für beliebigen planaren Graphen $G$.
  \item I.S.: Alle weiteren planaren Graphen lassen sich nur durch folgende Operationen aus dem im I.A. gegebenen Graphen konstruieren :
  	\begin{enumerate}
  		\item Hinzufügen eines Knotens, welcher über eine neue Kante mit dem Rest des Graphen verbunden ist. Damit steigt sowohl $|V|$ als auch $|E|$ um genau 1. Die Anzahl
  		der Facetten $|F|$ bleibt gleich. Damit bleibt die Gültigkeit der Euler Formel erhalten (beide Seiten der Gleichung erhöhen sich um genau 1). 
  		\item Hinzufügen einer Kante, die bereits bestehende Knoten verbindet. Die Anzahl
  		der Knoten $|V|$ bleibt somit gleich. Die Anzahl der Kanten $|E|$ steigt um 1. 
  		Die Anzahl der Facetten $|F|$ steigt ebenfalls um genau 1, da durch das Verbinden
  		der bereits vorhandenen Knoten eine "`alte"' Facette in zwei "`neue"' Facetten
  		(eine zusätzliche) aufgeteilt wird (Die rechte Seite der Gleichung bleibt gleich).
  		
	\end{enumerate}  
	Damit ist die Gültigkeit der Euler Formel für beliebige planare Graphen gezeigt. \checkmark		  
\end{itemize}

\subsection*{(b) Zeigen Sie, dass jede Triangulierung einer Menge von $n$ Punkten, von denen $r$ extrem sind, genau $2 (n - 1) - r$ Dreiecke enthält.}

Beweis durch strukturelle Induktion (Annahme: Keine zwei Punkte $p_i$, $p_j$ sind identisch):

\begin{itemize}
	\item I.A. (n = 3)): Es gilt $2 (n - 1) - r = 2 (3 - 1) - 3 = 4 - 3 = 1$ . \checkmark
	
	(Im initialen Fall mit  n = 3 Punkten sind alle 3 Punkte extrem (also $r = 3$) und die Triangulierung liefert genau 1 Dreieck) 
	\item I.V.: $2 (n - 1) - r$ gelte für eine beliebige Triangulierung mit $\geq 3$ Punkten.
	\item I.S.: Alle weiteren Triangulierungen ($> 3$ Punkte) lassen sich nur durch folgende Operationen/Fälle aus der im I.A. gegebenen Triangulierung konstruieren :
	\begin{enumerate}
		\item \textbf{Hinzufügen eines neuen Punktes, welcher neuer Extrempunkt ist:}
			Hier muss noch gezeigt werden, dass genau ein zusätzliches Dreieck entsteht.
			???????
		\item \textbf{Hinzufügen eines neuen Punktes, welcher kein neuer Extrempunkt ist:}
			Der neue Punkt liegt, wenn er kein neuer Extrempunkt ist, in einem der 
			bereits bestehenden Dreiecke $d$ der Triangulierung. $d$ lässt sich aufgrund
			des neuen Punktes nun in 3 neue Dreiecke aufteilen. D.h. es kommen 2 neue Dreiecke hinzu. Dies entspricht dem Verhalten 	der Gleichung $2 (n - 1) - r$. Denn in diesem Fall wird $n$ um 1 erhöht, während $r$ gleich bleibt. D.h. es werden zwei zusätzliche Dreiecke ausgegeben. \checkmark
	\end{enumerate}
\end{itemize}


\section*{Aufgabe 3 - Voronoi-Diagramm}


Wir haben ein Voronoi Diagramm $VD(S)$ gegeben, wobei die Datenstruktur (DS) von den Voronoi Kanten (VK) einer Voronoi Region (VR) in einem doppelt verketteten Liste $ZG_{p_i}$ geordnet seien. Beachte das Head und Tail von $ZG$ verbunden sind. D.h. auch dass die unendlichen Kanten einer offenen Facette im Graphen benachbart sind und die Zugriffszeit auf die Nachbarn in konstanter Zeit möglich sind.

Algorithmus:

\begin{itemize}

\item Wir suchen den Punkt $p_i$ aus $S$ mit den kleinsten x- und y-Wert. Der Punkt $p_i$ ist offensichtlich in der konvexen Hülle. $p_i$ kann in $\mathcal{O}(n)$ Zeit berechnet werden.

\item Nun betrachten wir vom Voronoi Punkt $p_i$ die VR $p_i$ und suchen eine unendliche Kante $e$ in der Datenstruktur (DS) $ZG_{p_i}$. Die Anzahl der Vergleiche kann in Worst-Case nicht mehr $3 n-6$ beanspruchen, da es die maximale Anzahl von Kanten in einem VD entspricht. Das liegt weiterhin in $\mathcal{O}(n)$.

\item Die gegenüberliegende VR $p_j$ von der Kante $e$ wird in die konvexe Hülle aufgenommen.

\item Es werden die Vorgänger und Nachfolger Kante von $e$ im $ZG_{p_j}$ gesucht, einer der beiden Kanten muss eine unendliche Kante $e'$ sein. Das Finden der benachbarten VR die eine offen Facette ist kann somit in $\mathcal{O}(1)$ Zeit bestimmt werden. 

\item Füge die VR $p_{j+1}$ von $e'$ zu konvexen Hülle hinzu und setze $e = e'$ wie $j=j+1$. Führe den letzten Schritt solange aus bis $j=i$ ist. Dies kann in $\mathcal{O}(n)$ Zeit beanspruchen.

\end{itemize}

Also läuft der Algorithmus in $\mathcal{O}(n)$.


\end{document}
