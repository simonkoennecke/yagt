%Dokumententyp
\documentclass[a4paper]{article}


\usepackage[a4paper,left=2cm, right=3cm, top=2cm]{geometry}

%Kodierung
\usepackage[utf8]{inputenc}
\usepackage[T1]{fontenc}

%Grafiken einbinden
\usepackage{graphicx}

%Position von Grafiken und Tabellen erzwingen:
\usepackage{float}

%URLs im Literaturverzeichnis
\usepackage{url}

\usepackage{amsmath}

%Schriftart Arial:
% \usepackage{helvet}

%Figures with text around it:
\usepackage{wrapfig}

\usepackage{listings}

%seitennummern rechts:
% \usepackage{fancyhdr}
% \fancyhf{} % clear all header and footers
% \renewcommand{\headrulewidth}{0pt} % remove the header rule
% \rfoot{\thepage}
% \fancypagestyle{plain}{%redefining plain pagestyle
% \fancyhf %clear all headers and footers fields
% \fancyhead[R]{\thepage} %prints the page number on the right side of the header
% }

%Schriftart Times New Roman "like"
\usepackage{txfonts}

%Sprache
\usepackage[german]{babel}


\usepackage{listings}
\usepackage{color}
\definecolor{javared}{rgb}{0.6,0,0} % for strings
\definecolor{javagreen}{rgb}{0.25,0.5,0.35} % comments
\definecolor{javapurple}{rgb}{0.5,0,0.35} % keywords
\definecolor{javadocblue}{rgb}{0.25,0.35,0.75} % javadoc
 
\lstset{language=Java,
basicstyle=\ttfamily,
keywordstyle=\color{javapurple}\bfseries,
stringstyle=\color{javared},
commentstyle=\color{javagreen},
morecomment=[s][\color{javadocblue}]{/**}{*/},
numbers=left,
numberstyle=\tiny\color{black},
stepnumber=1,
numbersep=5pt,
tabsize=4,
showspaces=false,
lineskip={-1.5pt},
showstringspaces=false}

%Tabellenextras
\usepackage{tabularx}

%Zeilenabstand 1.5
\linespread{1.5}
\usepackage{setspace}

%Figure Captions mit Fußnoten
\usepackage{footnote}
%\setlength{\parindent}{0pt} 


%itemize items richtig ausrichten (nicht links überlappen!)
% \setlist{leftmargin=0}

% %%%%TITELSEITE%%%%%%(
% \title{ Konzept und Implementierung\\ eines Systems zur \\Anforderung und Verwaltung von virtuellen privaten Clustern}
% \author{\textbf{\large Bachelorarbeit}}
% 
% \date{zur Erlangung des akademischen Grades Bachelor of Science an der Universität Paderborn im Fachbereich Informatik im Studiengang Bachelor Informatik}

% %%%%TITELSEITE%%%%%%)

% \pagestyle{fancy}
\begin{document}

\title{Algorithmische Geometrie - Sommersemester 2015\\
       3. Aufgabenblatt }
\author{Simon Koennecke und Felix Bröker}
\date{}
\maketitle

\section*{Aufgabe 1 - Bewegungsplanung in der Ebene}

Gegeben sind die Hindernisse $H = \left\{p_1, \dots, p_n\right\}$, der Start $s$, das Ziel $t$ und der Radius $r$ des Roboters. Des Weiteren sind der Start- und Endpunkt im Radius des Roboters frei von Hindernissen. Die Voronoi Kanten (VK) seien die einzig möglichen Wege für den Roboter. D.h. ein großzügiges Umfahren ist mit dem Algorithmus nicht möglich.

\begin{itemize}

\item Berechne $VD(H \cup \left\{s,t\right\})$ ohne die VK ins Unendliche.

\item Prüfe für alle VK $e$ und die Punkte der angrenzenden Voronoi-Regionen $p_i$ und $p_j$, ob gilt: $d(p_i, p_j) \geq r$. In diesem Fall ist sichergestellt, dass der Roboter die Kante passieren kann, ohne mit einem der Hindernisse $p_i$ oder $p_j$ zu kollidieren. Sollte jedoch $d(p_i, p_j) < r$ gelten, so entferne $e$ aus $VD$. 

Nach Voraussetzung brauchen
die Kanten der Region $s$ und $t$ hier nicht berücksichtigt werden. 

\item Erstelle einen Graphen $G = (V,E)$, wobei $E$
der Menge der übrig gebliebenen Voronoi-Kanten entspricht.
Das Kantengewicht einer jeden Kante $e$ des Graphen $G$
wird auf den Wert $1$ festgelegt. Die Knotenmenge $V$ wird so konstruiert, dass der Startpunkt $q_s$ alle Voronoi-Kanten der Voronoi-Region $s$ vereint. Dasselbe Verfahren wird analog für den Zielpunkt $q_t$ angewandt. Alle anderen Voronoi-Konten VK werden direkt in $V$ übernommen. 

Der Graph $G$ kann zyklisch und nicht zusammenhängend sein.

\item Suche in dem Graphen $G$ den Weg $w$ zwischen $q_s$ und $q_t$. Hier kann ein Shortest Path Algorithmus angewendet werden, z.B. der Algorithmus von Dijkstra.

\item Geben den Weg $w$ aus. Sollte $w$ kein Weg sein, existiert kein kollisionsfreier Pfad mit den gegebenen Hindernissen von $s$ nach $t$.

\end{itemize}  

\section*{Aufgabe 2 - Geometrische Graphen}

\subsection*{(a) Beweisen Sie Eulers Formel}

https://www.ics.uci.edu/~eppstein/junkyard/euler/

Eulers Formel:

\textit{Sei G ein planarer, zusammenhängender Graph, dann gilt:}
 $$\underbrace{|V|}_{Knoten} = \underbrace{|E|}_{Kanten} - \underbrace{|F|}_{Facetten} + 2 $$

Beweis durch Induktion über die Anzahl der Knoten:

\begin{itemize}
  \item I.A.: ( $|V| = 1$ )
  \item I.V.: $|V| = |E| - |F| + 2 $ gelte für beliebiges $|V| \in \mathbb{N}$ 
  \item I.S.: ( $|V| \leadsto |V| + 1$ ):  
\end{itemize}

\subsection*{(b) Zeigen Sie, dass jede Triangulierung einer Menge von $n$ Punkten, von denen $r$ extrem sind, genau $2 (n - 1) - r$ Dreiecke enthält.}



\section*{Aufgabe 3 - Voronoi-Diagramm}


Wir haben ein Voronoi Diagramm $VD(S)$ gegeben, wobei die Voronoi Regionen (VR) und deren Voronoi Kanten (VK) in einen zyklischen Graphen $ZG_{p_i}$ geordnet sind. Die unendlichen Kanten einer offenen Facette sind benachbart im Graphen. 

Algorithmus:

\begin{itemize}

\item Wir suchen den Punkt $p_i$ aus $S$ mit den kleinsten x- und y-Wert. Der Punkt $p_i$ ist offensichtlich in der konvexen Hülle. $p_i$ kann in $\mathcal{O}(n)$ Zeit berechnet werden.

\item Nun wird von der VR $p_i$ eine unendliche Kante $e$ im Graph $ZG_{p_i}$ gesucht. Die Anzahl der Vergleiche kann in Worst-Case nicht mehr $3 n-6$ beanspruchen. Das liegt weiterhin in $\mathcal{O}(n)$.

\item Die benachbarte VR $p_j$ von der Kante $e$ wird in die konvexe Hülle aufgenommen.

\item Suche den Vorgänger und Nachfolger Kante von $e$ im $ZG_{p_j}$, einer der beiden Kanten muss eine unendliche Kante $e'$ sein. Das Finden der VR ist in $\mathcal{O}(1)$ Zeit möglich, wenn die Datenstruktur im $ZG$ eine konstante Zugriffszeit auf Vor- und Nachfolger möglich ist. 

\item Füge die VR $p_{j+1}$ von $e'$ zu konvexen Hülle hinzu und setze $e = e'$ wie $j=j+1$. Führe den letzten Schritt solange aus bis $j=i$ ist. Dies kann in $\mathcal{O}(n)$ Zeit beanspruchen.

\end{itemize}

Also läuft der Algorithmus in $\mathcal{O}(n)$.


\end{document}
