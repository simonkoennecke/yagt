%Dokumententyp
\documentclass[a4paper]{article}


\usepackage[a4paper,left=2cm, right=3cm, top=2cm]{geometry}

%Kodierung
\usepackage[utf8]{inputenc}
\usepackage[T1]{fontenc}

%Grafiken einbinden
\usepackage{graphicx}

%Position von Grafiken und Tabellen erzwingen:
\usepackage{float}

%URLs im Literaturverzeichnis
\usepackage{url}

\usepackage{amsmath}

%Schriftart Arial:
% \usepackage{helvet}

%Figures with text around it:
\usepackage{wrapfig}

\usepackage{listings}

%seitennummern rechts:
% \usepackage{fancyhdr}
% \fancyhf{} % clear all header and footers
% \renewcommand{\headrulewidth}{0pt} % remove the header rule
% \rfoot{\thepage}
% \fancypagestyle{plain}{%redefining plain pagestyle
% \fancyhf %clear all headers and footers fields
% \fancyhead[R]{\thepage} %prints the page number on the right side of the header
% }

%Schriftart Times New Roman "like"
\usepackage{txfonts}

%Sprache
\usepackage[german]{babel}


\usepackage{listings}
\usepackage{color}
\definecolor{javared}{rgb}{0.6,0,0} % for strings
\definecolor{javagreen}{rgb}{0.25,0.5,0.35} % comments
\definecolor{javapurple}{rgb}{0.5,0,0.35} % keywords
\definecolor{javadocblue}{rgb}{0.25,0.35,0.75} % javadoc
 
\lstset{language=Java,
basicstyle=\ttfamily,
keywordstyle=\color{javapurple}\bfseries,
stringstyle=\color{javared},
commentstyle=\color{javagreen},
morecomment=[s][\color{javadocblue}]{/**}{*/},
numbers=left,
numberstyle=\tiny\color{black},
stepnumber=1,
numbersep=5pt,
tabsize=4,
showspaces=false,
lineskip={-1.5pt},
showstringspaces=false}

%Tabellenextras
\usepackage{tabularx}

%Zeilenabstand 1.5
\linespread{1.5}
\usepackage{setspace}

%Figure Captions mit Fußnoten
\usepackage{footnote}
%\setlength{\parindent}{0pt} 


%itemize items richtig ausrichten (nicht links überlappen!)
% \setlist{leftmargin=0}

% %%%%TITELSEITE%%%%%%(
% \title{ Konzept und Implementierung\\ eines Systems zur \\Anforderung und Verwaltung von virtuellen privaten Clustern}
% \author{\textbf{\large Bachelorarbeit}}
% 
% \date{zur Erlangung des akademischen Grades Bachelor of Science an der Universität Paderborn im Fachbereich Informatik im Studiengang Bachelor Informatik}

% %%%%TITELSEITE%%%%%%)

% \pagestyle{fancy}
\begin{document}

\title{Algorithmische Geometrie - Sommersemester 2015\\
       3. Aufgabenblatt }
\author{Simon Koennecke und Felix Bröker}
\date{}
\maketitle

\section*{Aufgabe 1 - Bewegungsplanung in der Ebene}

Gegen sind die Hindernisse $H = \left\{p_1, \dots, p_n\right\}$, der Start $s$, das Ziel $t$ vom Roboter und den Radius $r$ des Roboters. Wir gehen davon aus, dass die Voronoi Kanten die einzig möglichen Wege sind. D. h. ein Umfahren ist nicht möglich.

\begin{itemize}

\item Berechne $VD(H \cup \left\{s,t\right\})$ ohne die Voronoi Kanten ins unendliche.

\item Prüfe ob die umliegenden Kanten von $s$ und $t$ mehr als $r$ Abstand zu den Hindernissen haben. Sollte das nicht der Fall sein wäre keine Navigation vom Start oder zum Ziel möglich.

\item Prüfe für alle Voronoi Kanten $e$, ob die Kante zu den Hindernissen $p_i$ und $p_j$ an der kleinsten Stelle mehr als $r$ Abstand haben. Sollte das nicht der Fall sein wird die Kante aus $VD$ entfernt.

\item Wähle für den Start ein Voronoi Knoten $q_s$ aus so dass der Knoten $q_s$ an der Voronoi Region $s$ liegt.

\item Wähle für das Ziel ein Voronoi Knoten $q_t$ aus so dass der Knoten $q_t$ an der Voronoi Region $t$ liegt.

\item Stelle ein Graphen $G$ mit allen Voronoi Kanten als $E$ und mit allen Voronoi Knoten $V$ auf.

\item Suche in den Graphen $G$ den Weg $w$ zwischen $q_s$ und $q_t$.

\item Geben den Weg $w$ aus. Sollte $w$ kein Weg sein gibt es kein Pfad mit den gegebenen Hindernissen von $s$ nach $t$.

\end{itemize}  

\section*{Aufgabe 2 - Geometrische Graphen}

\subsection*{(a) Beweisen Sie Eulers Formel}

https://www.ics.uci.edu/~eppstein/junkyard/euler/

\subsection*{(b) Zeigen Sie, dass jede Triangulierung einer Menge von $n$ Punkten, von denen $r$ extrem sind, genau $2 (n - 1) - r$ Dreiecke enthält.}



\section*{Aufgabe 3 - Voronoi-Diagramm}



\end{document}
