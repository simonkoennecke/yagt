%Dokumententyp
\documentclass[a4paper]{article}

\usepackage[a4paper,left=2cm, right=3cm, top=2cm]{geometry}

%Kodierung
\usepackage[utf8]{inputenc}
\usepackage[T1]{fontenc}

%Grafiken einbinden
\usepackage{graphicx}
\usepackage{subfigure} 

%Position von Grafiken und Tabellen erzwingen:
\usepackage{float}

%URLs im Literaturverzeichnis
\usepackage{url}

\usepackage{amsmath}

%Vektoren einfacher angeben:
\newcommand{\vektor}[1]{\left( \begin{array}{c} #1 \end{array} \right) }


%Schriftart Arial:
% \usepackage{helvet}

%Figures with text around it:
\usepackage{wrapfig}

\usepackage{listings}

%seitennummern rechts:
% \usepackage{fancyhdr}
% \fancyhf{} % clear all header and footers
% \renewcommand{\headrulewidth}{0pt} % remove the header rule
% \rfoot{\thepage}
% \fancypagestyle{plain}{%redefining plain pagestyle
% \fancyhf %clear all headers and footers fields
% \fancyhead[R]{\thepage} %prints the page number on the right side of the header
% }

%Schriftart Times New Roman "like"
\usepackage{txfonts}

%Sprache
\usepackage[german]{babel}

%Checkmarks: (usage: \checkmark)
\usepackage{dingbat}

\usepackage{listings}
\usepackage{color}
\definecolor{javared}{rgb}{0.6,0,0} % for strings
\definecolor{javagreen}{rgb}{0.25,0.5,0.35} % comments
\definecolor{javapurple}{rgb}{0.5,0,0.35} % keywords
\definecolor{javadocblue}{rgb}{0.25,0.35,0.75} % javadoc
 
\lstset{language=Java,
basicstyle=\ttfamily,
keywordstyle=\color{javapurple}\bfseries,
stringstyle=\color{javared},
commentstyle=\color{javagreen},
morecomment=[s][\color{javadocblue}]{/**}{*/},
numbers=left,
numberstyle=\tiny\color{black},
stepnumber=1,
numbersep=5pt,
tabsize=4,
showspaces=false,
lineskip={-1.5pt},
showstringspaces=false}

%Tabellenextras
\usepackage{tabularx}

%Zeilenabstand 1.5
\linespread{1.5}
\usepackage{setspace}

%Figure Captions mit Fußnoten
\usepackage{footnote}
%\setlength{\parindent}{0pt} 


%itemize items richtig ausrichten (nicht links überlappen!)
% \setlist{leftmargin=0}

% %%%%TITELSEITE%%%%%%(
% \title{ Konzept und Implementierung\\ eines Systems zur \\Anforderung und Verwaltung von virtuellen privaten Clustern}
% \author{\textbf{\large Bachelorarbeit}}
% 
% \date{zur Erlangung des akademischen Grades Bachelor of Science an der Universität Paderborn im Fachbereich Informatik im Studiengang Bachelor Informatik}

% %%%%TITELSEITE%%%%%%)

% \pagestyle{fancy}
\begin{document}

\title{Algorithmische Geometrie - Sommersemester 2015\\
       7. Aufgabenblatt }
\author{Simon Koennecke und Felix Bröker}
\date{}
\maketitle

\section*{Aufgabe 1 - Anfragezeit bei $k$d-Bäumen}
Um die Anfragezeit der orthogonalen Bereichssuche für einen 3-dimensionalen $k$d-Baum zu analysieren, 
beschränken wir uns zunächst auf die Betrachtung einer Anfrage mit 
der Eingabe einer Achsen-parallelen Linie (statt z.B. eines gesamten Quaders).
Dazu betrachten wir die vom Algorithmus besuchten Knoten (ausgenommen der dazwischen liegenden
Punkte) und stellen in Abhängigkeit des aktuellen Knotens folgende Rekursionsgleichungen auf:

\begin{align}
 Q(n) &= 1 + 1 Q(\frac{n}{2}) &| kte Dimension\\
 Q(n) &= 1 + 2 Q(\frac{n}{2}) &| (k-1)te-Dimension\\
 Q(n) &= 1 + 2 Q(\frac{n}{2}) &| (k-2)te-Dimension\\ 
\end{align}

Diese Gleichungen ergeben sich aus der genannten Bereichsabfrage, welche
orthogonal zur $k$ten Dimension ist. 

Wir fassen im Weiteren 3 Ebenen des Baums zusammen und erhalten durch Einsetzen folgende Rekursionsgleichung:

\begin{align}
Q(n) = 4 Q(\frac{n}{8}) + \mathcal{O}(1) \\
Q(n) = 4^i + \sum_{a=0}^{i-1} 2^a \mathcal{O}(1), i = \log_8 n
\end{align}

Aus der Gleichung 5 lässt sich die verallgemeinerte Form (Gleichung 6) ableiten. 



Wir können den ersten Summand der Gleichung (6) mit $4^{log_8(n)} = 2^2^{\frac{log_2(n)}{log_2(8)}} =  n^{\frac{2}{3}}$ abschätzen.
Der zweite Summand $\sum_{a=0}^{\log_8 n -1} 2^a$ lässt sich über die geometrische Summe mit $\mathcal{O}(n^{\frac{2}{3}})$ abschätzen. Damit erhält man insgesamt $2 \mathcal{O}(n^{\frac{2}{3}}) = \mathcal{O}(n^{\frac{2}{3}})$.

Die Laufzeit für eine der genannten orthogonalen Linien kann wie gezeigt mit $\mathcal{O}(n^{\frac{2}{3}})$ 
abgeschätzt werden. Da im 3-dimensionalen Fall der angefragte Quader 12 Kanten besitzt, erhalten wir
im schlimmsten Fall eine Laufzeitabschätzung mit $12*\mathcal{O}(n^{\frac{2}{3}}) = \mathcal{O}(n^{\frac{2}{3}})$.


\end{document}

