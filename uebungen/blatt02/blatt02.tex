%Dokumententyp
\documentclass[a4paper]{article}


\usepackage[a4paper,left=2cm, right=3cm, top=2cm]{geometry}

%Kodierung
\usepackage[utf8]{inputenc}
\usepackage[T1]{fontenc}

%Grafiken einbinden
\usepackage{graphicx}

%Position von Grafiken und Tabellen erzwingen:
\usepackage{float}

%URLs im Literaturverzeichnis
\usepackage{url}

\usepackage{amsmath}

%Schriftart Arial:
% \usepackage{helvet}

%Figures with text around it:
\usepackage{wrapfig}

\usepackage{listings}

%seitennummern rechts:
% \usepackage{fancyhdr}
% \fancyhf{} % clear all header and footers
% \renewcommand{\headrulewidth}{0pt} % remove the header rule
% \rfoot{\thepage}
% \fancypagestyle{plain}{%redefining plain pagestyle
% \fancyhf %clear all headers and footers fields
% \fancyhead[R]{\thepage} %prints the page number on the right side of the header
% }

%Schriftart Times New Roman "like"
\usepackage{txfonts}

%Sprache
\usepackage[german]{babel}


\usepackage{listings}
\usepackage{color}
\definecolor{javared}{rgb}{0.6,0,0} % for strings
\definecolor{javagreen}{rgb}{0.25,0.5,0.35} % comments
\definecolor{javapurple}{rgb}{0.5,0,0.35} % keywords
\definecolor{javadocblue}{rgb}{0.25,0.35,0.75} % javadoc
 
\lstset{language=Java,
basicstyle=\ttfamily,
keywordstyle=\color{javapurple}\bfseries,
stringstyle=\color{javared},
commentstyle=\color{javagreen},
morecomment=[s][\color{javadocblue}]{/**}{*/},
numbers=left,
numberstyle=\tiny\color{black},
stepnumber=1,
numbersep=5pt,
tabsize=4,
showspaces=false,
lineskip={-1.5pt},
showstringspaces=false}

%Tabellenextras
\usepackage{tabularx}

%Zeilenabstand 1.5
\linespread{1.5}
\usepackage{setspace}

%Figure Captions mit Fußnoten
\usepackage{footnote}
%\setlength{\parindent}{0pt} 


%itemize items richtig ausrichten (nicht links überlappen!)
% \setlist{leftmargin=0}

% %%%%TITELSEITE%%%%%%(
% \title{ Konzept und Implementierung\\ eines Systems zur \\Anforderung und Verwaltung von virtuellen privaten Clustern}
% \author{\textbf{\large Bachelorarbeit}}
% 
% \date{zur Erlangung des akademischen Grades Bachelor of Science an der Universität Paderborn im Fachbereich Informatik im Studiengang Bachelor Informatik}

% %%%%TITELSEITE%%%%%%)

% \pagestyle{fancy}
\begin{document}

\title{Algorithmische Geometrie - Sommersemester 2015\\
       2. Aufgabenblatt }
\author{Simon Koennecke und Felix Bröker}
\date{}
\maketitle

\section*{Aufgabe 1 - Rotating Calipers}

Der wichtigeste Algorihmus in unserem Programm ist die Prozedur $calcAntipodal$. Diese Prozedur berechnet alle Antipodal Punkte von einem konvexen Polygon.

\begin{lstlisting}
    public void calcAntipodal() {
        initAntiPodalList();
        findExtrema();

        int i = mMaxPt, j = mMinPt, cnt = 0;
        //Vec[0-1] are current caliper
        //Vec[2-3] are current edges of the polygon
        Point[] vec = new Point[]{
                new Point2D(1, 0),
                new Point2D(-1, 0),
                getPoint(getSuccessor(i)).subtract(getPoint(i)),
                getPoint(getSuccessor(j)).subtract(getPoint(j))
        };
        //angle[0]: Angle between upper caliper and edge
		//angle[1]: Angle between lower caliper and edge
        double[] angle = new double[]{
                vec[2].angle(vec[0]),
                vec[3].angle(vec[1])
        };

        do {
            //On which Edge flash the caliper first?
            if (angle[0] > angle[1]) {
                //Lower Bound flash first
                j = getSuccessor(j);
                //move caliper
                vec[1] = vec[3];
                vec[0] = vec[3].multiply(-1);
                //goto next edge
                vec[3] = getPoint(getSuccessor(j)).subtract(getPoint(j));
            } else {
                //Upper Bound flash first
                i = getSuccessor(i);
                //move caliper
                vec[0] = vec[2];
                vec[1] = vec[2].multiply(-1);
                //goto next edge
                vec[2] = getPoint(getSuccessor(i)).subtract(getPoint(i));
            }
            //Add new index as a antipodal pair
            addAntiPodal(i, j);
			//Calc new angle between the current selected edge and caliper
            angle[0] = vec[2].angle(vec[0]);
            angle[1] = vec[3].angle(vec[1]);
        } while (i != mMinPt || j != mMaxPt);
    }

\end{lstlisting}
Das finden des maximalen Durchmessers wird mit dem betrachten aller Antipodal Paare ermöglicht. Die Berechnung könnte mit konstanten mehr Aufwand auch im Zuge der Ausführung der Funktion $calcAntipodal$ geschehen. Wir haben uns wegen der Übersichtlichkeit dagegen Entschieden und nehmen den Mehraufwand das wiederholte Durchgehens aller Antipodalen Paare in Kauf. Die Funktion $getDiameter$ übernimmt die Berechnung, wobei $calcAntipodal$ im Vorfeld ausgeführt werden muss.

\begin{lstlisting}
	public double getDiameter(){
        double max = -1, tmp;
        for (int i = 0; i < mAntipodal.size(); i++) {
            for (int j = 0; j < mAntipodal.get(i).size(); j++) {
                tmp = getPoint(i).distance(getPoint(mAntipodal.get(i).get(j)));
                if (max < tmp) {
                    max = tmp;
                    mMaxAntipodal = new int[]{i, mAntipodal.get(i).get(j)};
                }
            }
        }
        return max;
    }
\end{lstlisting}

\section*{Aufgabe 2 - Konvexe Hülle}

Die Aufgabenstellung beschrieben zwei disjunkte und konvexe Polygone $S_1$ und $S_2$. Die die Punkte der Polygone $S_i$ seien im Uhrzeigersinn geordnet, $S_i \subseteq S \subseteq \mathcal{R}^2$, $i = \left\{1, 2\right\}$ und $|S| = n$ gilt. Des Weiteren kann man die Punkte $s_j \in S_1, 0 \leq j \leq |S_1|$ und $t_j \in S_2, 0 \leq j \leq |S_2|$ in konstanter Zeit addressieren, sprich in $\mathcal{O}(1)$.

\subsection*{(a) - Zeigen Sie, dass das Finden der Brücken in $\mathcal{O}(\log n)$ Zeit möglich ist.}

Es gilt zu zeigen, dass das Finden der Brücke zwischen zwei disjunkte und konvexen Polygonen in $\mathcal{O}(\log n)$ Zeit möglich ist. Dazu nehmen wir an, dass die Extrema $s_{y-min}$, $t_{y-min}$ und $s_{y-max}$, $t_{y-max}$ bekannt sein. Die Extrema kann man in im Divide-and-Conquer-Verfahren bei jeden Merg-Schritt in konstanter Zeit berechnen. Wir wählen ein Polygon aus, in unserem Fall $S_1$. Es wird eine Brücke vom Punkt $s_{y-min}$ (hier genannt $s_k$) zu $t_j$ gespannt. Dabei soll der Winkel maximal werden: $max \left\{\sphericalangle \left(\overline{s_k,s_{k+1}},\overline{s_k,t_j}\right)\right\}, 0 \leq j \leq |S_2|$. Es wird nach einem $j$ gesucht so das der Winkel maximal ist. Da auf die Elemente $t_j$ konstant zugegriffen werden kann die Binärsuche angewendet werden. Die Laufzeit der Binärsuche ist bekannt, nämlich $\mathcal{O}(\log n)$. Da wir nicht wissen in welcher Richtung die Winkel größer werden, werden beide Richtungen abgelaufen. Dieses Prozedere wird ebenfalls mit $s_{y-max}$ durch gespielt. Daraus resultieren zwei Brücken $\overline{s_{y-min}, t_r}$ und $\overline{s_{y-max}, t_p}$ die in einer Laufzeit von $\mathcal{O}(4 \cdot \log n)$ gefunden werden kann. Diese Laufzeit liegt wiederum in $\mathcal{O}(\log n)$.

Quelle: http://www.cs.ubc.ca/~snoeyink/papers/nosep.ps.gz, Stand 03.05.2015 um 17:23 Uhr. 

\subsection*{(b) - Analysieren Sie die Laufzeit des gesamten Algorithmus, getrennt nach der fürs Sortieren am Anfang einerseits und dem rekursiven Teil andererseits. }

Die Beschreibung des Algorithmus erfolgt in drei Schritten:

\begin{enumerate}

\item Sortiere $S$ anhand der $x$-Koordinate.

\item Teile $S$ ($|S| = k$) in $S_1 = \left\{s_1, ..., s_{\lfloor \frac{k}{2} \rfloor } \right\}$ und $S_2 = \left\{s_{ \lceil \frac{k}{2} \rceil}, ..., s_n \right\}$ solange auf bis $|S| \leq 3$ ist.

\item Führe die Polygone $S_1$ und $S_2$ zusammen, so das es ein konvex Polygon bildet.

\end{enumerate}

Der erste Schritt können wir mit dem Merg-Sort realisieren, die Laufzeit ist uns bekannt mit $\mathcal{O}(n \cdot \log n)$ wie die Rekursionsgleichung $T(n) = T(\lfloor \frac{n}{2} \rfloor) + T(\lceil \frac{n}{2} \rceil) + d$.
Der zweite Schritt benötigt in etwa $\lceil \log n \rceil$ aufrufe. Der dritte Schritt beöntigt $\mathcal{O}(\log n)$ Zeit, dies wissen wir aus der vorherigen Teilaufgabe. Die Laufzeit vom dritten Schritt nimmt den Platz von $d$ bei Rekursionsgleich ein. Also erhalten wir für unseren Algorithmus folgende Rekursionsgleich:

\begin{equation}
	T(1) = T(2) = T(3) = 1 	\linebreak
\end{equation}

\begin{equation}
	T(n) = T(\lfloor \frac{n}{2} \rfloor) + T(\lceil \frac{n}{2} \rceil) + \log n \linebreak
\end{equation}
Sei $n=2^k$.

\begin{equation}
T(n) = 2 \cdot T(\frac{n}{2}) + \log n \linebreak
\end{equation}

\begin{equation}
T(2^k) = 2 \cdot T(\frac{2^{k}}{2}) + \log 2^k = 2 \cdot T(2^{k-1}) + k
\end{equation}

\begin{equation}
T(2^k) = 2 \cdot \left( 2 \cdot T(2^{k-2}) + k \right) + k
\end{equation}
Wie schon bekannt wird der zweite Schritt $\lceil \log n \rceil$ ausgeführt. Daher wird der rekursive Aufruf von $T(2^k)$ nur $\log n$ mal ausgeführt. Daraus können wir folgendes Schließen:
\begin{equation}
1 + \sum_{i=0}^{\log n} 2^i \cdot k \rightarrow 1 + k \cdot (2^{\log_2 n + 1} - 1) \rightarrow 1 + k \cdot (2 \cdot n - 1)
\end{equation}
Wir setzen für $k = log(n)$ wegen der oben getroffenen Annahme $n = 2^k$.
\begin{equation}
1 + \log n \cdot (2 \cdot n - 1) \rightarrow 1 + 2 \cdot n \cdot \log n - \log n
\end{equation}
Also liegt der Algorithmus in $\mathcal{O}(n \cdot \log n)$.

\end{document}