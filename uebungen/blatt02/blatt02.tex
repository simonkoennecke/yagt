%Dokumententyp
\documentclass[a4paper]{article}


\usepackage[a4paper,left=2cm, right=3cm, top=2cm]{geometry}

%Kodierung
\usepackage[utf8]{inputenc}
\usepackage[T1]{fontenc}

%Grafiken einbinden
\usepackage{graphicx}

%Position von Grafiken und Tabellen erzwingen:
\usepackage{float}

%URLs im Literaturverzeichnis
\usepackage{url}

\usepackage{amsmath}

%Schriftart Arial:
% \usepackage{helvet}

%Figures with text around it:
\usepackage{wrapfig}

\usepackage{listings}

%seitennummern rechts:
% \usepackage{fancyhdr}
% \fancyhf{} % clear all header and footers
% \renewcommand{\headrulewidth}{0pt} % remove the header rule
% \rfoot{\thepage}
% \fancypagestyle{plain}{%redefining plain pagestyle
% \fancyhf %clear all headers and footers fields
% \fancyhead[R]{\thepage} %prints the page number on the right side of the header
% }

%Schriftart Times New Roman "like"
\usepackage{txfonts}

%Sprache
\usepackage[german]{babel}


\usepackage{listings}
\usepackage{color}
\definecolor{javared}{rgb}{0.6,0,0} % for strings
\definecolor{javagreen}{rgb}{0.25,0.5,0.35} % comments
\definecolor{javapurple}{rgb}{0.5,0,0.35} % keywords
\definecolor{javadocblue}{rgb}{0.25,0.35,0.75} % javadoc
 
\lstset{language=Java,
basicstyle=\ttfamily,
keywordstyle=\color{javapurple}\bfseries,
stringstyle=\color{javared},
commentstyle=\color{javagreen},
morecomment=[s][\color{javadocblue}]{/**}{*/},
numbers=left,
numberstyle=\tiny\color{black},
stepnumber=1,
numbersep=5pt,
tabsize=4,
showspaces=false,
lineskip={-1.5pt},
showstringspaces=false}

%Tabellenextras
\usepackage{tabularx}

%Zeilenabstand 1.5
\linespread{1.5}
\usepackage{setspace}

%Figure Captions mit Fußnoten
\usepackage{footnote}
%\setlength{\parindent}{0pt} 


%itemize items richtig ausrichten (nicht links überlappen!)
% \setlist{leftmargin=0}

% %%%%TITELSEITE%%%%%%(
% \title{ Konzept und Implementierung\\ eines Systems zur \\Anforderung und Verwaltung von virtuellen privaten Clustern}
% \author{\textbf{\large Bachelorarbeit}}
% 
% \date{zur Erlangung des akademischen Grades Bachelor of Science an der Universität Paderborn im Fachbereich Informatik im Studiengang Bachelor Informatik}

% %%%%TITELSEITE%%%%%%)

% \pagestyle{fancy}
\begin{document}

\title{Algorithmische Geometrie - Sommersemester 2015\\
       2. Aufgabenblatt }
\author{Simon Koennecke und Felix Bröker}
\date{}
\maketitle

\section*{Aufgabe 1 - Rotating Calipers}

Der wichtigeste Algorihmus in unserem Programm ist die Prozedur $calcAntipodal$. Diese Prozedur berechnet alle Antipodal Punkte von einem konvexen Polygon.

\begin{lstlisting}
    public void calcAntipodal() {
        initAntiPodalList();
        findExtrema();

        int i = mMaxPt, j = mMinPt, cnt = 0;
        //Vec[0-1] are current caliper
        //Vec[2-3] are current edges
        Point[] vec = new Point[]{
                new Point2D(1, 0),
                new Point2D(-1, 0),
                getPoint(getSuccessor(i)).subtract(getPoint(i)),
                getPoint(getSuccessor(j)).subtract(getPoint(j))
        };
        //angle[0]: Angle between upper caliper and edge
		//angle[1]: Angle between lower caliper and edge
        double[] angle = new double[]{
                vec[2].angle(vec[0]),
                vec[3].angle(vec[1])
        };

        do {
            //On which Edge flash the caliper first?
            if (angle[0] > angle[1]) {
                //Lower Bound flash first
                j = getSuccessor(j);
                //move caliper
                vec[1] = vec[3];
                vec[0] = vec[3].multiply(-1);
                //goto next edge
                vec[3] = getPoint(getSuccessor(j)).subtract(getPoint(j));
            } else {
                //Upper Bound flash first
                i = getSuccessor(i);
                //move caliper
                vec[0] = vec[2];
                vec[1] = vec[2].multiply(-1);
                //goto next edge
                vec[2] = getPoint(getSuccessor(i)).subtract(getPoint(i));
            }
            //Add new index as a antipodal pair
            addAntiPodal(i, j);
			//Calc new angle between the current selected edge and caliper
            angle[0] = vec[2].angle(vec[0]);
            angle[1] = vec[3].angle(vec[1]);
        } while (i != mMinPt || j != mMaxPt);
    }

\end{lstlisting}

Das finden des maximalen Durchmesser wird mit dem durch gehen aller Antipodal ermöglicht. Dies könnte auch im Zuge des berechnen von der Funktion $calcAntipodal$ geschehen. Wir haben uns wegen der Übersichtlichkeit dagegen Entschieden und nehmen den mehr Aufwand in kauf und haben die Funktion $getDiameter$ wie folgt definiert:

\begin{lstlisting}
	public double getDiameter(){
        double max = -1, tmp;
        for (int i = 0; i < mAntipodal.size(); i++) {
            for (int j = 0; j < mAntipodal.get(i).size(); j++) {
                tmp = getPoint(i).distance(getPoint(mAntipodal.get(i).get(j)));
                if (max < tmp) {
                    max = tmp;
                    mMaxAntipodal = new int[]{i, mAntipodal.get(i).get(j)};
                }
            }
        }
        return max;
    }
\end{lstlisting}

\section*{Aufgabe 2 - Konvexe Hülle}

\subsection*{(a) - Zeigen Sie, dass das Finden der Brücken in $\mathcal{O}(\log n)$ Zeit möglich ist.}

Die Aufgabenstellung beschrieben zwei disjunkte und konvexe Polygone $S_1$ und $S_2$. Die die Punkte der Polygone $S_i$ seien im Uhrzeigersinn geordnet, $S_i \subseteq S \subseteq \mathcal{R}^2$, $i = \left\{1, 2\right\}$ und $|S| = n$ gilt. Des Weiteren kann man die Punkte $s_j \in S_1, 0 \leq j \leq |S_1|$ und $t_j \in S_2, 0 \leq j \leq |S_2|$ in konstanter Zeit addressieren, sprich in $\mathcal{O}(1)$.

Es gilt zu zeigen, dass das Finden der Brücke zwischen zwei disjunkte und konvexen Polygonen in $\mathcal{O}(\log n)$ Zeit möglich ist. Dazu nehmen wir an, dass die Extrema $s_{y-min}$, $t_{y-min}$ und $s_{y-max}$, $t_{y-max}$ bekannt sein. Die Extrema kann man in im Divide-and-Conquer-Verfahren bei jeden Merg-Schritt in konstanter Zeit berechnen. Wir wählen ein Polygon aus, in unserem Fall $S_1$. Es wird eine Brücke vom Punkt $s_{y-min}$ (hier genannt $s_k$) zu $t_j$ gespannt. Dabei soll der Winkel maximal werden: $max \left\{angle\left(\overline{s_k,s_{k+1}},\overline{s_k,t_j}\right)\right\}, 0 \leq j \leq |S_2|$. Es wird nach einem $j$ gesucht so das der Winkel maximal ist. Da auf die Elemente $t_j$ konstant zugegriffen werden kann die Binärsuche angewendet werden. Die Laufzeit der Binärsuche ist bekannt, nämlich $\mathcal{O}(\log n)$. Da wir nicht wissen in welcher Richtung die Winkel größer werden, werden beide Richtungen abgelaufen. Dieses Prozedere wird ebenfalls mit $s_{y-max}$ durch gespielt. Daraus resultieren zwei Brücken $\overline{s_{y-min}, t_r}$ und $\overline{s_{y-max}, t_p}$ die in einer Laufzeit von $\mathcal{O}(4 \cdot \log n)$ gefunden werden kann. Diese Laufzeit liegt wiederum in $\mathcal{O}(\log n)$.

Quelle: http://www.cs.ubc.ca/~snoeyink/papers/nosep.ps.gz, Stand 03.05.2015 um 17:23 Uhr. 

\subsection*{(b) - Analysieren Sie die Laufzeit des gesamten Algorithmus, getrennt nach der fürs 
Sortieren am Anfang einerseits und dem rekursiven Teil andererseits. }

\end{document}