%Dokumententyp
\documentclass[a4paper]{article}

\usepackage[a4paper,left=2cm, right=3cm, top=2cm]{geometry}

%Kodierung
\usepackage[utf8]{inputenc}
\usepackage[T1]{fontenc}

%Grafiken einbinden
\usepackage{graphicx}
\usepackage{subfigure} 

%Position von Grafiken und Tabellen erzwingen:
\usepackage{float}

%URLs im Literaturverzeichnis
\usepackage{url}

\usepackage{amsmath}

%Vektoren einfacher angeben:
\newcommand{\vektor}[1]{\left( \begin{array}{c} #1 \end{array} \right) }


%Schriftart Arial:
% \usepackage{helvet}

%Figures with text around it:
\usepackage{wrapfig}

\usepackage{listings}

%seitennummern rechts:
% \usepackage{fancyhdr}
% \fancyhf{} % clear all header and footers
% \renewcommand{\headrulewidth}{0pt} % remove the header rule
% \rfoot{\thepage}
% \fancypagestyle{plain}{%redefining plain pagestyle
% \fancyhf %clear all headers and footers fields
% \fancyhead[R]{\thepage} %prints the page number on the right side of the header
% }

%Schriftart Times New Roman "like"
\usepackage{txfonts}

%Sprache
\usepackage[german]{babel}

%Checkmarks: (usage: \checkmark)
\usepackage{dingbat}

\usepackage{listings}
\usepackage{color}
\definecolor{javared}{rgb}{0.6,0,0} % for strings
\definecolor{javagreen}{rgb}{0.25,0.5,0.35} % comments
\definecolor{javapurple}{rgb}{0.5,0,0.35} % keywords
\definecolor{javadocblue}{rgb}{0.25,0.35,0.75} % javadoc
 
\lstset{language=Java,
basicstyle=\ttfamily,
keywordstyle=\color{javapurple}\bfseries,
stringstyle=\color{javared},
commentstyle=\color{javagreen},
morecomment=[s][\color{javadocblue}]{/**}{*/},
numbers=left,
numberstyle=\tiny\color{black},
stepnumber=1,
numbersep=5pt,
tabsize=4,
showspaces=false,
lineskip={-1.5pt},
showstringspaces=false}

%Tabellenextras
\usepackage{tabularx}

%Zeilenabstand 1.5
\linespread{1.5}
\usepackage{setspace}

%Figure Captions mit Fußnoten
\usepackage{footnote}
%\setlength{\parindent}{0pt} 


%itemize items richtig ausrichten (nicht links überlappen!)
% \setlist{leftmargin=0}

% %%%%TITELSEITE%%%%%%(
% \title{ Konzept und Implementierung\\ eines Systems zur \\Anforderung und Verwaltung von virtuellen privaten Clustern}
% \author{\textbf{\large Bachelorarbeit}}
% 
% \date{zur Erlangung des akademischen Grades Bachelor of Science an der Universität Paderborn im Fachbereich Informatik im Studiengang Bachelor Informatik}

% %%%%TITELSEITE%%%%%%)

% \pagestyle{fancy}
\begin{document}

\title{Algorithmische Geometrie - Sommersemester 2015\\
       8. Aufgabenblatt }
\author{Simon Koennecke und Felix Bröker}
\date{}
\maketitle

\section*{Aufgabe 1 - Vorverarbeitungszeit für Bereichsbäume}

Im folgenden sehen wir uns zunächst in Pseudocode an, 
wie ein 2d-Bereichsbaum aufgebaut werden kann:

\begin{lstlisting}
Eingabe: pX, pY (nach x- bzw. y-Koordinate sortiertes Array der Punkte aus P)
Ausgabe: Wurzelknoten des 2d-Bereichbaums

Create-2d-Range-Tree(pX, pY)
IF size(pX) == 0
	return NULL;
ELSE
	yTree = Create-1d-Range-Tree(pX, pY);

IF size(pX) == 1
	return new Node(pX[0], yTree, NULL, NULL);
ELSE
	leftNode  = Create-2d-Range-Tree(pX[0..size(pX)/2], pY);
	rightNode = Create-2d-Range-Tree(pX[size(pX)/2+1..size(pX)-1], pY);
	return new Node(pX[size(pX)/2], yTree, leftNode, rightNode);
\end{lstlisting}

Die 4 Parameter von "`Node"' geben dabei den zu speichernden x-Wert (der jeweilige Median), 
den zum Knoten zugehörigen 1d-Bereichsbaum (bzgl. der y-Koordinaten der zugehörigen Punkte),
den linken, sowie rechten Kindknoten an. "`Create-1d-Range-Tree(pX, pY)"' erstellt einen 1d-Bereichsbaum
für alle Punkte in pX nach deren y-Koordinate. 

\subsection*{Laufzeitabschätzung}
Zur Laufzeitabschätzung stellen wir folgende Rekursionsgleichungen auf:
$$T(\leq 1) = \mathcal{O}(1)$$
Im Fall von $n \leq 1$ (Rekursionsanker) haben wir eine konstante Laufzeit.
Dies können wir auch an obigem Pseudocode ablesen. Ist die Eingabegröße $ n < 1$ bzw. $n = 0$
wird lediglich ein "`NULL"'-Element zurückgegeben. Im Fall $n = 1$ wird lediglich ein neuer Knoten
und dessen 1-elementiger 1d-Range-Tree (ebenfalls konstante Zeit) angelegt und zurückgegeben. 
In allen anderen Fällen setzt sich die Laufzeit aus der Zeit für die Konstruktion eines 1d-Range-Trees
(Zeile 8), der Zeit für den Aufruf beider Rekursionen (Zeilen 13,14) und der Erstellung des Rückgabeknotens (Zeile 15) zusammen:

$$T(n) = T(\lfloor n/2\rfloor) + T(\lceil n/2\rceil) + \underbrace{\mathcal{O}(n) + \mathcal{O}(1)}_{\mathcal{O}(n)}$$

\textit{TODO: Hier noch begründen, weshalb wir den yTree, also 1d-Range-Tree mithilfe der 
sortierten Liste pY in O(n) Zeit konstruieren können..}

 Die Zeit für die Erstellung des Rückgabeknotens inklusive anderer
konstanter Operationen wie z.B. der verwendeten IF-Abfragen kann mit $\mathcal{O}(1)$ abgeschätzt
werden. Da die aktuelle Punktemenge der Größe $n$ anhand ihres Medians geteilt und als Eingabe der
nächsten Rekursionen verwendet werden, und nicht sicher ist, ob $n$ eine 2er-Potenz ist, werden
jeweils die auf- bzw. abgerundeten Hälften der Eingabegröße zur Abschätzung ($T(\lfloor n/2\rfloor)$  und $T(\lceil n/2\rceil)$) verwendet. 

Die aufgestellte Rekursionsgleichung entspricht genau der Rekursionsgleichung wie wir sie von Sortieralgorithmen wie z.B. Mergesort kennen. Für Mergesort ist die Laufzeitabschätzung 
$T(n) = \mathcal{O}(n \log n)$ bekannt. Die hier vorgestellte Berechnung/Konstruktion eines
2d-Range-Trees benötigt also ebenfalls $\mathcal{O}(n \log n)$ Zeit.

\section*{Aufgabe 2 - dynamische Segmentbäume}
\section*{Aufgabe 3 - Punkt-Rechteck-Anfragen}


\end{document}

