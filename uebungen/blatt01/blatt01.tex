%Dokumententyp
\documentclass[a4paper]{article}


\usepackage[a4paper,left=2cm, right=3cm, top=2cm]{geometry}

%Kodierung
\usepackage[utf8]{inputenc}
\usepackage[T1]{fontenc}

%Grafiken einbinden
\usepackage{graphicx}

%Position von Grafiken und Tabellen erzwingen:
\usepackage{float}

%URLs im Literaturverzeichnis
\usepackage{url}

\usepackage{amsmath}

%Schriftart Arial:
% \usepackage{helvet}

%Figures with text around it:
\usepackage{wrapfig}

\usepackage{listings}

%seitennummern rechts:
% \usepackage{fancyhdr}
% \fancyhf{} % clear all header and footers
% \renewcommand{\headrulewidth}{0pt} % remove the header rule
% \rfoot{\thepage}
% \fancypagestyle{plain}{%redefining plain pagestyle
% \fancyhf %clear all headers and footers fields
% \fancyhead[R]{\thepage} %prints the page number on the right side of the header
% }

%Schriftart Times New Roman "like"
\usepackage{txfonts}

%Sprache
\usepackage[german]{babel}


%Tabellenextras
\usepackage{tabularx}

%Zeilenabstand 1.5
\linespread{1.5}
\usepackage{setspace}

%Figure Captions mit Fußnoten
\usepackage{footnote}
%\setlength{\parindent}{0pt} 


%itemize items richtig ausrichten (nicht links überlappen!)
% \setlist{leftmargin=0}

% %%%%TITELSEITE%%%%%%(
% \title{ Konzept und Implementierung\\ eines Systems zur \\Anforderung und Verwaltung von virtuellen privaten Clustern}
% \author{\textbf{\large Bachelorarbeit}}
% 
% \date{zur Erlangung des akademischen Grades Bachelor of Science an der Universität Paderborn im Fachbereich Informatik im Studiengang Bachelor Informatik}

% %%%%TITELSEITE%%%%%%)

% \pagestyle{fancy}
\begin{document}

\title{Algorithmische Geometrie - Sommersemester 2015\\
       1. Aufgabenblatt }
\author{Simon Koennecke (4420162) und Felix Bröker (4725990)}
\date{}
\maketitle

\section*{Aufgabe 1 - Geradendarstellung}

\subsection*{(a) - Umrechnung zwischen verschiedenen Darstellungen} 
\begin{description}
 \item[affine Hülle $\rightarrow$ Nullstellenmenge:] 
 Der Normalenvektor $\vec{n}$ der Geraden ist mit den Komponenten des Richtungsvektors $\vec{r} = (\vec{q} - \vec{p})$
 als $\vec{n} = (-r_2, r_1)$ gegeben. Damit sind die Parameter $a = -r_2$ und $b = r_1$  der Nullstellenmenge bereits bestimmt.
 Der Parameter $c$ lässt sich dann mittels der Komponenten des Punktes $p$ der affinen Hülle wie folgt berechnen: $c = -1 * (p_1a + p_2b)$.
 
 \item[affine Hülle $\rightarrow$ Hessesche Normalform:] Um die Hessesche Normalform zu erhalten, wird ähnlich
 wie zuvor beschrieben vorgegangen. Der Normalenvektor $\vec{n_r} = (-r_2, r_1)$ muss hier jedoch noch wegen $||\vec{n}|| = 1$ wie folgt 
 normiert werden: $\vec{n} = \vec{n_r}/||\vec{n_r}||$, bevor $c$ berechnet wird. 
 
 \item[Nullstellenmenge $\rightarrow$ affine Hülle:]
 Da der Normalenvektor $\vec{n} = (a,b)$ bereits durch die Nullstellenmengen-Darstellung gegeben ist, lässt sich der Richtungsvektor
 der affinen Hülle $\vec{r}$
 einfach mittels $\vec{r} = (b,-a)$ beschreiben. Als Stützvektor $\vec{p}$ kommt jeder beliebige Punkt auf der Geraden in Frage. Dazu bestimmt man ein
 beliebiges $x \in \mathbb{R}$ und berechnet das zugehörige $y$ mittels der gegebenen linearen Gleichung wie folgt: $y = \frac{-c -ax}{b}$.
 
 \item[Nullstellenmenge $\rightarrow$ Hessesche Normalform:]
 Ähnlich wie im vorherigen Fall ist der Normalenvektor bereits mit $\vec{n_r} = (a,b)$ gegeben, muss jedoch noch wegen $||\vec{n}|| = 1$ wie folgt 
 normiert werden: $\vec{n} = \vec{n_r}/||\vec{n_r}||$. Weiterhin nehme gültigen Punkt $\vec{x}=(x_1,x_2)$ aus der Nullstellenmengen-Darstellung und berechne
 hiermit $c$ wie folgt: $c = -1 * (n_1x_1 + n_2x_2)$.
 
 \item[Hessesche Normalform $\rightarrow$ Nullstellenmenge:]
 Die Parameter $a$ und $b$ sind bereits durch den Normalenvektor $\vec{n}$ gegeben. Das $c$ ist ebenfalls bekannt. 
 Eine alternative Interpretation der Hesseschen Normalform reicht somit aus. Es sind keine arithmetischen Operationen notwendig. 
 
 \item[Hessesche Normalform $\rightarrow$ affine Hülle:]
 Mit der im vorherigen Schritt beschriebenen Interpretation lässt sich die affine Hülle wie in "`\textbf{Nullstellenmenge $\rightarrow$ affine Hülle}"'
 berechnen/bestimmen. 
 
\end{description}



\subsection*{(b)}

Sei gegeben $p \in \mathbb{R}^2$ und $g = \left\{x \in \mathbb{R}^2| nx + c = 0 \right\}$. Des Weiteren nehmen wir an, dass $|c|$ der Abstand von Koordinatenursprung und diese Annahme sei bewiesen.

Sei $g'$ eine Grade die durch $p$ geht und die parallel zu $g$ verläuft, somit kann man $g'$ wie folt beschreiben wird: $g' = \left\{x \in \mathbb{R}^2| nx + c = \pm \delta \right\}$. Sei $d$ die Abstandfunktion. Dann kann man die Verhältnisse von $p$, $g$ und $g'$ wie folgt beschreiben: $d\left(p,g)\right) = d\left(g',g)\right) = \delta$. Daraus kann man schliessen $\delta = | \pm \left(nx+c\right)| = |nx+c|$.


\section*{Aufgabe 2 - Implementierung}

Siehe Ausgabe des Programms.

Die zwei wichtigsten Funktionen sind $intersection$ von zwei Graden und die Funktion $isInSegment$ bei Schnittpunkte von Strecken. 

\begin{lstlisting}
public Point intersection(OrientatedLine2D line) {
        // We need two points of each line.
        Point p1 = getLocationVector();
        Point p2 = getAnotherPointOnLine();
        Point p3 = line.getLocationVector();
        Point p4 = line.getAnotherPointOnLine();

        //Please see for the formula here:
        //http://en.wikipedia.org/wiki/Line%E2%80%93line_intersection#Given_two_points_on_each_line

        //When this is zero the both lines are parallel.
        if (((p1.getX() - p2.getX())*(p3.getY() - p4.getY())) - ((p1.getY() - p2.getY()) * (p3.getX() - p4.getX())) == 0) {
            return null;
        }

        double x, y, z;
        //Divider
        z = ((p1.getX() - p2.getX()) * (p3.getY() - p4.getY())) -
                ((p1.getY() - p2.getY()) * (p3.getX() - p4.getX()));

        x = (((p1.getX() * p2.getY()) - (p1.getY() * p2.getX())) * (p3.getX() - p4.getX())) -
                (p1.getX() - p2.getX()) * (p3.getX() * p4.getY() - p3.getY() * p4.getX());
        x /= z;
        y = (p1.getX() * p2.getY() - p1.getY() * p2.getX()) * (p3.getY() - p4.getY()) -
                (p1.getY() - p2.getY()) * (p3.getX() * p4.getY() - p3.getY() * p4.getX());
        y /= z;
        return new Point2D(x, y);
    }
\end{lstlisting}

\begin{lstlisting}
public boolean isInSegment(Point pt) {
        return (mStart.distance(pt) + pt.distance(mEnd)) == mStart.distance(mEnd);
}
\end{lstlisting}

\section*{Aufgabe 3 - BHD}

Gegeben sei ein n-Eck $P$ in BHD. Es soll entschieden werden, ob $ p \in P$ und $g \cap P$ ist, wobei $g$ eine beliebigen Grade ist. $P$ wird hier als Menge aufgefasst. Wir wissen das der BHD eine Höhe von $log(n)$ hat. Der BHD kann in Level $P_0, P_1, ... P_k$ aufgeteilt werden und das n-Eck ist im Uhrzeigersinn mit $s_1, ..., s_n$ nummeriert.

\begin{description}
	\item[Es soll gezeigt werden, dass $p \in P$ in $log(n)$ entschieden werden kann:]

\begin{itemize}

\item Schritt 1: Überprüfe ob $p$ gleich $s_1$ ist. Falls ja: gilt $p \in P$. Sonst weiter beim zweiten Schritt mit $i=1$.

\item Schritt 2: Sei $p \in P_i$, wenn $min(x_{s_i}, x_{s_i+1}) \leq p_x \leq max(x_{s_i}, x_{s_i+1})$ und $min(y_{s_i}, y_{s_i+1}) \leq p_x \leq max(y_{s_i}, y_{s_i+1})$. Sonst weiter mit dem dritten Schritt.

\item Schritt 3: Entscheide ob es im rechten oder linken Halbraum von $P_i$ liegt. Beim rechten Halbraum gehe $s_i$ Endpunkt der Strecke, also gehe zum zweiten Schritt mit $i= i-1$. Beim linken Halbraum $s_i$ der Startpunkt der Strecke ist also gehe zum zweiten Schritt mit $i= i+1$.

\end{itemize}




	\item[Es soll gezeigt werden, dass $g \cap	 P$ in $log(n)$ entschieden werden kann:]

\end{description}

\end{document}