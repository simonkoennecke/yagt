%Dokumententyp
\documentclass[a4paper]{article}


\usepackage[a4paper,left=2cm, right=3cm, top=2cm]{geometry}

%Kodierung
\usepackage[utf8]{inputenc}
\usepackage[T1]{fontenc}

%Grafiken einbinden
\usepackage{graphicx}

%Position von Grafiken und Tabellen erzwingen:
\usepackage{float}

%URLs im Literaturverzeichnis
\usepackage{url}

\usepackage{amsmath}

%Schriftart Arial:
% \usepackage{helvet}

%Figures with text around it:
\usepackage{wrapfig}

\usepackage{listings}

%seitennummern rechts:
% \usepackage{fancyhdr}
% \fancyhf{} % clear all header and footers
% \renewcommand{\headrulewidth}{0pt} % remove the header rule
% \rfoot{\thepage}
% \fancypagestyle{plain}{%redefining plain pagestyle
% \fancyhf %clear all headers and footers fields
% \fancyhead[R]{\thepage} %prints the page number on the right side of the header
% }

%Schriftart Times New Roman "like"
\usepackage{txfonts}

%Sprache
\usepackage[german]{babel}


\usepackage{listings}
\usepackage{color}
\definecolor{javared}{rgb}{0.6,0,0} % for strings
\definecolor{javagreen}{rgb}{0.25,0.5,0.35} % comments
\definecolor{javapurple}{rgb}{0.5,0,0.35} % keywords
\definecolor{javadocblue}{rgb}{0.25,0.35,0.75} % javadoc
 
\lstset{language=Java,
basicstyle=\ttfamily,
keywordstyle=\color{javapurple}\bfseries,
stringstyle=\color{javared},
commentstyle=\color{javagreen},
morecomment=[s][\color{javadocblue}]{/**}{*/},
numbers=left,
numberstyle=\tiny\color{black},
stepnumber=1,
numbersep=5pt,
tabsize=4,
showspaces=false,
lineskip={-1.5pt},
showstringspaces=false}

%Tabellenextras
\usepackage{tabularx}

%Zeilenabstand 1.5
\linespread{1.5}
\usepackage{setspace}

%Figure Captions mit Fußnoten
\usepackage{footnote}
%\setlength{\parindent}{0pt} 


%itemize items richtig ausrichten (nicht links überlappen!)
% \setlist{leftmargin=0}

% %%%%TITELSEITE%%%%%%(
% \title{ Konzept und Implementierung\\ eines Systems zur \\Anforderung und Verwaltung von virtuellen privaten Clustern}
% \author{\textbf{\large Bachelorarbeit}}
% 
% \date{zur Erlangung des akademischen Grades Bachelor of Science an der Universität Paderborn im Fachbereich Informatik im Studiengang Bachelor Informatik}

% %%%%TITELSEITE%%%%%%)

% \pagestyle{fancy}
\begin{document}

\title{Algorithmische Geometrie - Sommersemester 2015\\
       3. Aufgabenblatt }
\author{Simon Koennecke und Felix Bröker}
\date{}
\maketitle

\section*{Aufgabe 1 - Implementation des Graham-Scan Algorithmus}

Wir verwenden den unten stehende Comparator $GrahamComparator$ zum Sortieren der Koordinaten. Dabei ist $mCenter$ der Punkt mit der kleinsten y-Koordinate und sollten mehrere Punkte auf y-min liegen wird der mit dem geringsten x-Wert gewählt. $mCenter$ wird zu weiteren Berechnung als neuer Ursprung definiert. Die Sortierung verwendet das Vergleichskriterium  des Cotangens, also hier nur das Verhältnis von x- zu y-Koordinate ($\frac{x}{y}$). 
Der Fall y == 0 wird vom Komparator abgefangen. Der entsprechende Punkt wird vor jedem anderen Punkt
mit y != 0 einsortiert, da der Punkt aufgrund der Lage des $mCenter$-Punktes nur auf der positiven x-Achse liegen kann.

Punkte mit gleichem Cotangens werden nach der Distanz zum Ursprung einsortiert (Punkte mit kleinerer Distanz vor Punkten mit größerer Distanz). 

\begin{lstlisting}
public class GrahamComparator implements Comparator<Point> {
    private Point mCenter;

    public GrahamComparator(Point center) {
        mCenter = center;
    }
   
    public int compare(Point a, Point b) {
    	//Let 'center' be the new origin:
    	a = a.subtract(mCenter);
    	b = b.subtract(mCenter);
    	
    	// mind. 1 y-Koordinate == 0:
    	if (a.getY() == 0 || b.getY() == 0) {
    		if (a.getY() != 0 || b.getY() != 0) { //XOR
    		    //Da mCenter "linkestes/unterstes" Element, y == 0 nur auf x groesser 0 moeglich.
    			return (a.getY() == 0) ? -1 : 1;
    		} 
    		//Both == 0
    		return (a.getX() > b.getX()) ? 1 : -1;
    	}

    	// y-Koordinate in beiden Faellen != 0:
    	double d1 = a.getX() / a.getY(), d2 = b.getX() / b.getY();
    	
    	if (d1 == d2) {
    		return a.getX() * a.getX() + a.getY() * a.getY() < b.getX() * b.getX() + b.getY() * b.getY() ? -1 : 1;
    	}
        return (d1 < d2) ? 1 : -1;

    }
}
\end{lstlisting}

Der Graham Scan bekommt eine Menge an unsortierten Punkten übergeben und berechnet eine konvexe Hülle. Die Berechnung der konvexen Hülle wird in der Funktion $computeConvexHull$ vorgenommen.

\begin{lstlisting}
public class GrahamScan {
    private Set mSet;

    public GrahamScan(Set set) {
        mSet = set;
        mSet.sort(new GrahamComparator(set.getMinYExtrema()));
        computeConvexHull();
    }

    /**
     * Source: http://de.wikipedia.org/wiki/Graham_Scan#Pseudocode
     */
    private void computeConvexHull() {
        int i = 1;
        while (i < mSet.getAllPoints().size()) {
            if (determinant(
                    mSet.get(mSet.getPredecessor(i)),
                    mSet.get(mSet.getSuccessor(i)),
                    mSet.get(i)) < 0) {
                i++;
            } else {
                mSet.remove(i);
                i--;
            }
        }
    }

    /**
     * @return AB zu C: 0 = collinear, 0 < right, 0 > left
     */
    private static double determinant(Point a, Point b, Point c) {
        return (b.getX() - a.getX()) * (c.getY() - a.getY()) -
                    (c.getX() - a.getX()) * (b.getY() - a.getY());
    }

    public Polygon getPolygon() {
        return new Polygon(mSet.getAllPoints());
    }
}
\end{lstlisting}


\section*{Aufgabe 2 - Inkrementelle Konstruktion der konvexen Hülle}

Es sei eine beliebiges konvexes Polygon $S = CH(S) = \left\{p_i, \dots, p_n\right\} \subset \mathcal{R}^2$ und $|S| = n$ gegeben. Nun soll ein Punkt $q$ zu $S$ hinzugefügt werden. Dabei soll die konvexe Hülle ggf. angepasst werden, falls $q$ außerhalb von $CH(S)$ liegt. 

Weiterhin sei $s_m$ der Schwerpunkt von $S$, welcher in $\mathcal{O}(n)$ Zeit berechnet werden kann. $h_i$ sei die Halbgerade von $s_m$ durch $p_i$. Das Prüfen, ob ein Punkt rechts oder links von einer Halbgeraden liegt, ist in $\mathcal{O}(1)$ Zeit möglich (siehe Aufgabe 1, durch Berechnung der Determinanten).

\subsection*{Algorithmus}

\begin{enumerate}

\item Prüfe ob $q \notin CH(S)$, dann weiter mit 2, sonst fertig

\item Finde den Kegel beschrieben durch $K_{h_r,h_s}$ (wobei gilt: $h_s$ ist Nachfolger von $h_r$), in dem $q$ liegt.

\begin{itemize}
	\item[2.1] $q$ wird nun zu $S$ hinzugefügt, so dass $p_s$ Nachfolger und $p_r$ Vorgänger von $q$ ist.
	
	\item[2.2] entlang beider Richtungen in $S$ führe folgende Schritte (hier beschrieben im Uhrzeigersinn (gegen den Uhrzeigersinn ist analog zu verstehen)) ausgehend von $q$ aus:
	
	\begin{itemize}
		\item bestimme die Determinante $T(q, Nachfolger(p_s), p_s)$. Ist diese $>0$ ($p_s$ ist Teil der konvexen Hülle von $S$), ist man fertig.
		\item im anderen Fall entferne $p_s$ aus $S$ und fahre mit Schritt 2.2 und $p_s = Nachfolger(p_s)$ fort. 
	\end{itemize}

\end{itemize}

\end{enumerate}

Notizen:
Zur Verwaltung der Punkte von $S$ soll die Datenstruktur "`Dynamic Array"'
eingesetzt werden. 
Diese besitzt folgende armotisierte Laufzeiten:
\begin{itemize}
 \item Add: O(1)
 \item Remove: O(1)
 \item Find: O(1)
\end{itemize}

\subsection*{Laufzeit}
\begin{enumerate}
\item Der erste Schritt ist in $\mathcal{O}(n)$ Zeit berechenbar. (z.B. durch
Prüfen, ob der Punkt in keinem der rechten Halbräume der Geraden, beschrieben durch
zwei aufeinander folgende Punkte $pi$ und $p_{i+1}$ liegt)
\item Im zweiten Schritt kann der Kegel, in welchem der Punkt $q$ liegt, in Laufzeit
$\mathcal{O}(n)$ gefunden werden. (hier z.B. ähnlich zu Schritt 1 durch Vergleiche 
mit den verschiedenen Halbräumen der $n$ Halbgeraden $h_i$)
\item Der Punkt $q$ kann in Schritt 2.1 in Zeit $\mathcal{O}(1)$ zu $S$ hinzugefügt werden.
\item Der Schritt 2.2 benötigt pro Durchlauf (also pro Berechnung und Auswertung der jeweiligen
Determinante und eventuellem Entfernen des aktuellen Punktes $p_s$) konstante Zeit, also $\mathcal{O}(1)$. Insgesamt wird dieser Schritt pro Richtung grob abgeschätzt $< n$-mal häufig ausgeführt, für beide Richtungen also $< 2n$-mal. Damit hat der Schritt 2.2 insgesamt eine Laufzeit von $\mathcal{O}(n)$.
\end{enumerate}

Die gesamte Laufzeit des Algorithmus kann also mit $\mathcal{O}(n)$ + $\mathcal{O}(n)$ + $\mathcal{O}(1)$ + $\mathcal{O}(n)$ = $\mathcal{O}(n)$ abgeschätzt werden. 



\section*{Aufgabe 3 - Untere Schranke}



\end{document}
