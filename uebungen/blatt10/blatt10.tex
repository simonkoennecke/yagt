%Dokumententyp
\documentclass[a4paper]{article}

\usepackage[a4paper,left=2cm, right=3cm, top=2cm]{geometry}

%Kodierung
\usepackage[utf8]{inputenc}
\usepackage[T1]{fontenc}

%Grafiken einbinden
\usepackage{graphicx}
\usepackage{subfigure} 

%Position von Grafiken und Tabellen erzwingen:
\usepackage{float}

%URLs im Literaturverzeichnis
\usepackage{url}

\usepackage{amsmath}

%Vektoren einfacher angeben:
\newcommand{\vektor}[1]{\left( \begin{array}{c} #1 \end{array} \right) }


%Schriftart Arial:
% \usepackage{helvet}

%Figures with text around it:
\usepackage{wrapfig}

\usepackage{listings}

%seitennummern rechts:
% \usepackage{fancyhdr}
% \fancyhf{} % clear all header and footers
% \renewcommand{\headrulewidth}{0pt} % remove the header rule
% \rfoot{\thepage}
% \fancypagestyle{plain}{%redefining plain pagestyle
% \fancyhf %clear all headers and footers fields
% \fancyhead[R]{\thepage} %prints the page number on the right side of the header
% }

%Schriftart Times New Roman "like"
\usepackage{txfonts}

%Sprache
\usepackage[german]{babel}

%Checkmarks: (usage: \checkmark)
\usepackage{dingbat}

\usepackage{listings}
\usepackage{color}
\definecolor{javared}{rgb}{0.6,0,0} % for strings
\definecolor{javagreen}{rgb}{0.25,0.5,0.35} % comments
\definecolor{javapurple}{rgb}{0.5,0,0.35} % keywords
\definecolor{javadocblue}{rgb}{0.25,0.35,0.75} % javadoc
 
\lstset{language=Java,
basicstyle=\ttfamily,
keywordstyle=\color{javapurple}\bfseries,
stringstyle=\color{javared},
commentstyle=\color{javagreen},
morecomment=[s][\color{javadocblue}]{/**}{*/},
numbers=left,
numberstyle=\tiny\color{black},
stepnumber=1,
numbersep=5pt,
tabsize=4,
showspaces=false,
lineskip={-1.5pt},
showstringspaces=false}

%Tabellenextras
\usepackage{tabularx}

%Zeilenabstand 1.5
\linespread{1.5}
\usepackage{setspace}

%
\usepackage{cancel}

%Figure Captions mit Fußnoten
\usepackage{footnote}
%\setlength{\parindent}{0pt} 

%Graphen/Trees zeichnen:
\usepackage{tikz}
\newcommand*\circled[1]{\tikz[baseline=(char.base)]{
            \node[shape=circle,draw,inner sep=2pt] (char) {#1};}}


%itemize items richtig ausrichten (nicht links überlappen!)
% \setlist{leftmargin=0}

% %%%%TITELSEITE%%%%%%(
% \title{ Konzept und Implementierung\\ eines Systems zur \\Anforderung und Verwaltung von virtuellen privaten Clustern}
% \author{\textbf{\large Bachelorarbeit}}
% 
% \date{zur Erlangung des akademischen Grades Bachelor of Science an der Universität Paderborn im Fachbereich Informatik im Studiengang Bachelor Informatik}

% %%%%TITELSEITE%%%%%%)

% \pagestyle{fancy}
\begin{document}

\title{Algorithmische Geometrie - Sommersemester 2015\\
       10. Aufgabenblatt }
\author{Simon Koennecke und Felix Bröker}
\date{}
\maketitle

\section*{Aufgabe 1}
Gegeben sei eine Punktmenge $P = \{ (x,y) | x,y \in \mathcal{R} \} \subset \mathcal{R}^2$.
Es ist die maximale Anzahl von Punkten $p \in P$ gesucht, die auf einer gemeinsamen Geraden $g$ liegen.

Zur Lösung der Aufgabe formulieren wir die Problemstellung in eine duale Darstellung wie folgt um:
Gegeben sei eine Menge von Geraden $G = \{a*x -y = b | a,b \in \mathcal{R} \wedge (x,y) \in P \}$.
Es ist die maximale Anzahl von Geraden $g \in G$ gesucht, die sich in einem gemeinsamen Punkt $q$ schneiden.
Sind dieser Punkt $q = (q_a, q_b)$ und die in diesem Punkt schneidenden Geraden $g \in G$ ermittelt, lassen sich über
die Dualitätsabbildung:

\begin{align}
q = (q_a, a_b) &\longmapsto l: y = q_a * x - q_b\\
g:  a*x -y = b &\longmapsto p = (x,y)
\end{align}

die ursprünglich gesuchte Gerade und Punkte direkt ermitteln. 

Unser folgender Algorithmus arbeitet also auf der dualen Darstellung und bestimmt Punkt $q$:
\vspace*{1cm}

 Initialisiere einen AVL-Baum $H$ und Variable $MAX = 0$.
 (Jeder Knoten des Baums $H$ soll dabei sowohl einen Schnittpunkt $s$ 
 als auch einen Zähler $c$ speichern können)
 
 Für alle $g \in G$:
 \begin{itemize}
  \item Durchlaufe alle $g' \in G \setminus \{g\}$ 
  \begin{itemize}
   \item Betrachte die Geraden $g$ und $g'$
   \begin{itemize}
       \item Fall 1 ($g$ und $g'$ liegen aufeinander):
	Dieser Fall ist aufgrund der Konstruktion von $G$ nicht möglich.
       \item Fall 2 ($g$ und $g'$ liegen parallel zueinander):
	Fahre mit nächster Schleifeniteration fort.
       \item Fall 3 ($g$ und $g'$ haben einen Schnittpunkt $s$):
       
	Aktualisiere $H$ folgendermaßen:
	\begin{itemize}
	 \item Fall 1 ($s \in H$):
	 Erhöhe den Zähler $c$ von $s$ in $H$ um 1.
	 Sollte $c$ jetzt $> MAX$ sein, aktualisiere $MAX$ mit $MAX = c$.
	 \item Fall 2 ($s \notin H$):
	 Füge $s$ mit dem Zählerwert $c = 1$ zu $H$ hinzu.
	 Sollte $c$ jetzt $> MAX$ sein, aktualisiere $MAX$ mit $MAX = c$.
	\end{itemize}

   \end{itemize}
 
  \end{itemize}
 \end{itemize}
 Gebe Wert der Variablen $MAX$ aus.
 
 \subsection*{Laufzeit}
 Die Laufzeit lässt sich wie folgt abschätzen:
 
 \begin{itemize}
  \item Das Durchlaufen aller Geraden benötigt $\mathcal{O}(n)$ Zeit
    \item Das jeweilige Vergleichen mit allen anderen Geraden benötigt ebenfalls $\mathcal{O}(n)$ Zeit.
  \item Die Bestimmung des Falles bei Betrachtung der Geraden $g$ und $g'$ geht in $\mathcal{O}(1)$ Zeit.
  \item Die Fälle 1 und 2 können dabei ebenfalls in $\mathcal{O}(1)$ Zeit bearbeitet werden.
  \item Der 3. Fall benötigt $\mathcal{O}(\log n)$ Zeit. 
  
  Denn das Einfügen in den AVL-Baum ist bekanntermaßen in $\mathcal{O}(\log n)$ Zeit möglich.
  Das Aktualisieren des Zählers $c$ eines Schnittpunkts $s$ kann durch das Suchen von $s$ in $H$
  und anschließendes Inkrementieren des Zählers $c$ in $\mathcal{O}(\log n)$ durchgeführt werden.
  
 \end{itemize}
 
 Daraus resultiert eine Gesamtlaufzeit von $\mathcal{O}(n) * (\mathcal{O}(n) * (\underbrace{\mathcal{O}(1) + \mathcal{O}(\log n)}_{\mathcal{O}(\log n)})) = \mathcal{O}(n)$



\section*{Aufgabe 2}
\section*{Aufgabe 3}

\end{document}

