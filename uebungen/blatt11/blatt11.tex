%Dokumententyp
\documentclass[a4paper]{article}

\usepackage[a4paper,left=2cm, right=3cm, top=2cm]{geometry}

%Kodierung
\usepackage[utf8]{inputenc}
\usepackage[T1]{fontenc}

%Grafiken einbinden
\usepackage{graphicx}
\usepackage{subfigure} 

%Position von Grafiken und Tabellen erzwingen:
\usepackage{float}

%URLs im Literaturverzeichnis
\usepackage{url}

\usepackage{amsmath}

%Vektoren einfacher angeben:
\newcommand{\vektor}[1]{\left( \begin{array}{c} #1 \end{array} \right) }


%Schriftart Arial:
% \usepackage{helvet}

%Figures with text around it:
\usepackage{wrapfig}

\usepackage{listings}

%seitennummern rechts:
% \usepackage{fancyhdr}
% \fancyhf{} % clear all header and footers
% \renewcommand{\headrulewidth}{0pt} % remove the header rule
% \rfoot{\thepage}
% \fancypagestyle{plain}{%redefining plain pagestyle
% \fancyhf %clear all headers and footers fields
% \fancyhead[R]{\thepage} %prints the page number on the right side of the header
% }

%Schriftart Times New Roman "like"
\usepackage{txfonts}

%Sprache
\usepackage[german]{babel}

%Checkmarks: (usage: \checkmark)
\usepackage{dingbat}

\usepackage{listings}
\usepackage{color}
\definecolor{javared}{rgb}{0.6,0,0} % for strings
\definecolor{javagreen}{rgb}{0.25,0.5,0.35} % comments
\definecolor{javapurple}{rgb}{0.5,0,0.35} % keywords
\definecolor{javadocblue}{rgb}{0.25,0.35,0.75} % javadoc
 
\lstset{language=Java,
basicstyle=\ttfamily,
keywordstyle=\color{javapurple}\bfseries,
stringstyle=\color{javared},
commentstyle=\color{javagreen},
morecomment=[s][\color{javadocblue}]{/**}{*/},
numbers=left,
numberstyle=\tiny\color{black},
stepnumber=1,
numbersep=5pt,
tabsize=4,
showspaces=false,
lineskip={-1.5pt},
showstringspaces=false}

%Tabellenextras
\usepackage{tabularx}

%Zeilenabstand 1.5
\linespread{1.5}
\usepackage{setspace}

%URLs
\usepackage{url}

%
\usepackage{cancel}

%Figure Captions mit Fußnoten
\usepackage{footnote}
%\setlength{\parindent}{0pt} 

%Graphen/Trees zeichnen:
\usepackage{tikz}
\newcommand*\circled[1]{\tikz[baseline=(char.base)]{
            \node[shape=circle,draw,inner sep=2pt] (char) {#1};}}


%itemize items richtig ausrichten (nicht links überlappen!)
% \setlist{leftmargin=0}

% %%%%TITELSEITE%%%%%%(
% \title{ Konzept und Implementierung\\ eines Systems zur \\Anforderung und Verwaltung von virtuellen privaten Clustern}
% \author{\textbf{\large Bachelorarbeit}}
% 
% \date{zur Erlangung des akademischen Grades Bachelor of Science an der Universität Paderborn im Fachbereich Informatik im Studiengang Bachelor Informatik}

% %%%%TITELSEITE%%%%%%)

% \pagestyle{fancy}
\begin{document}

\title{Algorithmische Geometrie - Sommersemester 2015\\
       11. Aufgabenblatt }
\author{Simon Koennecke und Felix Bröker}
\date{}
\maketitle

\section*{Aufgabe 1: Konfliktecken}



\section*{Aufgabe 2: Randomisiert inkrementelle Konstruktionen}



\subsection*{(a) konvexe Hülle}

Es sei eine beliebiges Punktmenge $P = \left\{p_1, \dots, p_n\right\} \subset \mathcal{R}^2$ und $|P| = n$ in allgemeiner Lage gegeben. 
Die ersten ersten drei Punkte spannen ein Dreieck, also eine Konvexe Hülle $S$ bestehend aus $\{p_1, p_2, p_3\}$, auf. 
Der Punkt $s$ sei der Schwerpunkt dieses Dreiecks $\{p_1, p_2, p_3\}$ (Berechnung ist in konstanter Zeit möglich).
Nun sollen Punkte $p_i$ ($i > 3$) zu der konvexen Hülle $S$ hinzugefügt werden. 
Dabei soll die konvexe Hülle ggf. angepasst werden, falls $p_i$ außerhalb von $S$ liegt. 
$h_i$ sei die Halbgerade von $s$ durch $p_i$. Das Prüfen, ob ein Punkt rechts oder links von einer Halbgeraden liegt, ist in $\mathcal{O}(1)$ Zeit möglich.

\subsection*{Algorithmus}

\begin{enumerate}

\item Solange $p_i \in P = \varnothing$:

\begin{enumerate}

  \item Finde den Kegel beschrieben durch $K_{h_r,h_s}$ (wobei gilt: $h_s$ ist Nachfolger von $h_r$), in dem $p_i$ liegt.

  \begin{itemize}
	  \item[2.1] $p_i$ wird nun zu $S$ hinzugefügt, so dass $p_s$ Nachfolger und $p_r$ Vorgänger von $q$ ist.
	  
	  \item[2.2] entlang beider Richtungen in $S$ führe folgende Schritte (hier beschrieben im Uhrzeigersinn (gegen den Uhrzeigersinn ist analog zu verstehen)) ausgehend von $q$ aus:
	  
	  \begin{itemize}
		  \item bestimme die Determinante $T(q, Nachfolger(p_s), p_s)$. Ist diese $>0$ ($p_s$ ist Teil der konvexen Hülle von $S$), ist man fertig.
		  \item im anderen Fall entferne $p_s$ aus $S$ und fahre mit Schritt 2.2 und $p_s = Nachfolger(p_s)$ fort. 
	  \end{itemize}

  \end{itemize}
  
\end{enumerate}

\end{enumerate}

Notizen:
Zur Verwaltung der Punkte von $S$ soll die Datenstruktur "`Dynamic Array"'
eingesetzt werden. 
Diese besitzt folgende armotisierte Laufzeiten:
\begin{itemize}
 \item Add: O(1)
 \item Remove: O(1)
 \item Find: O(1)
\end{itemize}

\subsection*{Laufzeit}
\begin{enumerate}
\item Der erste Schritt ist in $\mathcal{O}(n)$ Zeit berechenbar. (z.B. durch
Prüfen, ob der Punkt in keinem der rechten Halbräume der Geraden, beschrieben durch
zwei aufeinander folgende Punkte $pi$ und $p_{i+1}$ liegt)
\item Im zweiten Schritt kann der Kegel, in welchem der Punkt $q$ liegt, in Laufzeit
$\mathcal{O}(n)$ gefunden werden. (hier z.B. ähnlich zu Schritt 1 durch Vergleiche 
mit den verschiedenen Halbräumen der $n$ Halbgeraden $h_i$)
\item Der Punkt $q$ kann in Schritt 2.1 in Zeit $\mathcal{O}(1)$ zu $S$ hinzugefügt werden.
\item Der Schritt 2.2 benötigt pro Durchlauf (also pro Berechnung und Auswertung der jeweiligen
Determinante und eventuellem Entfernen des aktuellen Punktes $p_s$) konstante Zeit, also $\mathcal{O}(1)$. Insgesamt wird dieser Schritt pro Richtung grob abgeschätzt $< n$-mal häufig ausgeführt, für beide Richtungen also $< 2n$-mal. Damit hat der Schritt 2.2 insgesamt eine Laufzeit von $\mathcal{O}(n)$.
\end{enumerate}

Die gesamte Laufzeit des Algorithmus kann also mit $\mathcal{O}(n)$ + $\mathcal{O}(n)$ + $\mathcal{O}(1)$ + $\mathcal{O}(n)$ = $\mathcal{O}(n)$ abgeschätzt werden. 


\subsection*{(b) Voronoi-Diagramm}

  

\end{document}

