%Dokumententyp
\documentclass[a4paper]{article}

\usepackage[a4paper,left=2cm, right=3cm, top=2cm]{geometry}

%Kodierung
\usepackage[utf8]{inputenc}
\usepackage[T1]{fontenc}

%Grafiken einbinden
\usepackage{graphicx}
\usepackage{subfigure} 

%Position von Grafiken und Tabellen erzwingen:
\usepackage{float}

%URLs im Literaturverzeichnis
\usepackage{url}

\usepackage{amsmath}

%Vektoren einfacher angeben:
\newcommand{\vektor}[1]{\left( \begin{array}{c} #1 \end{array} \right) }


%Schriftart Arial:
% \usepackage{helvet}

%Figures with text around it:
\usepackage{wrapfig}

\usepackage{listings}

%seitennummern rechts:
% \usepackage{fancyhdr}
% \fancyhf{} % clear all header and footers
% \renewcommand{\headrulewidth}{0pt} % remove the header rule
% \rfoot{\thepage}
% \fancypagestyle{plain}{%redefining plain pagestyle
% \fancyhf %clear all headers and footers fields
% \fancyhead[R]{\thepage} %prints the page number on the right side of the header
% }

%Schriftart Times New Roman "like"
\usepackage{txfonts}

%Sprache
\usepackage[german]{babel}

%Checkmarks: (usage: \checkmark)
\usepackage{dingbat}

\usepackage{listings}
\usepackage{color}
\definecolor{javared}{rgb}{0.6,0,0} % for strings
\definecolor{javagreen}{rgb}{0.25,0.5,0.35} % comments
\definecolor{javapurple}{rgb}{0.5,0,0.35} % keywords
\definecolor{javadocblue}{rgb}{0.25,0.35,0.75} % javadoc
 
\lstset{language=Java,
basicstyle=\ttfamily,
keywordstyle=\color{javapurple}\bfseries,
stringstyle=\color{javared},
commentstyle=\color{javagreen},
morecomment=[s][\color{javadocblue}]{/**}{*/},
numbers=left,
numberstyle=\tiny\color{black},
stepnumber=1,
numbersep=5pt,
tabsize=4,
showspaces=false,
lineskip={-1.5pt},
showstringspaces=false}

%Tabellenextras
\usepackage{tabularx}

%Zeilenabstand 1.5
\linespread{1.5}
\usepackage{setspace}

%
\usepackage{cancel}

%Figure Captions mit Fußnoten
\usepackage{footnote}
%\setlength{\parindent}{0pt} 

%Graphen/Trees zeichnen:
\usepackage{tikz}
\newcommand*\circled[1]{\tikz[baseline=(char.base)]{
            \node[shape=circle,draw,inner sep=2pt] (char) {#1};}}


%itemize items richtig ausrichten (nicht links überlappen!)
% \setlist{leftmargin=0}

% %%%%TITELSEITE%%%%%%(
% \title{ Konzept und Implementierung\\ eines Systems zur \\Anforderung und Verwaltung von virtuellen privaten Clustern}
% \author{\textbf{\large Bachelorarbeit}}
% 
% \date{zur Erlangung des akademischen Grades Bachelor of Science an der Universität Paderborn im Fachbereich Informatik im Studiengang Bachelor Informatik}

% %%%%TITELSEITE%%%%%%)

% \pagestyle{fancy}
\begin{document}

\title{Algorithmische Geometrie - Sommersemester 2015\\
       Wichtige Infos für die Klausur}
\date{}
\maketitle

\section*{Grundlegende Definitionen}

\subsection*{Punkte, Strecken, Geraden, Ebenen}
\subsubsection*{Hessesche Normalform}
Kann verwendet werden, um in konstanter Zeit zu überprüfen, ob ein Punkt auf der Geraden/Ebenen
liegt und auf welcher Seite links (Wert der Gleichung $<0$) oder
rechts (Wert der Gleichung $>0$) dieser Punkt liegt.

\subsubsection*{Konvexe Mengen (Polygone, Konvexe Hüllen, ...)}
\begin{itemize}
	\item Der Durchschnitt beliebig vieler konvexer Mengen ist konvex.
	\item Die Konvexe Hülle $CH(S)$ einer Menge $S$ von Punkten ist die kleinste konvexe Menge
	welche $S$ enthält
	\item Das Innere eines konvexen Polygons entspricht dem Schnitt der offenen Halbräume rechts
	aller durch die Kanten definierten Geraden
\end{itemize}

\section*{Grundlegende Fragestellungen}
\subsubsection*{Liegt ein Punkt $p$ innerhalb eines konvexen Polygons $P$?}
Wenn das konvexe Polygon als sortierte Liste der Kanten im Uhrzeigersinn gegeben ist:
Liegt $p$ immer z.B. im rechten Halbraum/rechts aller durch die Kanten definierten Geraden.
Dies benötigt pro Kante konstante und insgesamt lineare Zeit $\mathcal{O}(n)$.
\subsubsection*{Bestimme die Schnittmenge einer Geraden $g$ mit einem konvexen Polygon $P$}
Für jede einzelne Kante kann der Schnitt in konstanter Zeit bestimmt werden.
Somit benötigt die Bestimmung der Schnittmenge lineare Zeit $\mathcal{O}(n)$.
\subsubsection*{Verbesserung durch Balanciert hierarchische Darstellung}
Die obigen Operationen können, wenn häufig ausgeführt, besser mit BHD in logarithmischer
Zeit $\mathcal{O}(\log n)$ gelöst werden.
\subsubsection*{Schnitt konvexer Polygone}
\begin{itemize}
	\item Der Schnitt von zwei konvexen Polygonen mit $n$ und $m$ Ecken hat maximal $n + m$ Ecken.
	\item Wenn zwei Polygone $P$ und $Q$ in BHD-Darstellung gegeben sind, lässt sich in 
	$\mathcal{O}(\log( n + m ))$ Zeit feststellen ob $P \cap Q = \varnothing$
\end{itemize}
\section*{Planare Graphen}
Ein planarer Graph hat $n$ Ecken, $f$ Facetten und $e$ Kanten.
\begin{itemize}
	\item Eulersche Polyederformel: $n + f - e = 2$
	\item $e \leq 3n - 6$
	\item $f \leq 2n -4$
	\item $n \leq 2f -4$
\end{itemize}

\subsection*{Voronoi-Diagramme}
\begin{itemize}
	\item Dual zur entsprechenden Delaunay-Triangulierung
\end{itemize}

\subsubsection*{Algorithmen zur Konstruktion von Voronoi-Diagrammen}
\begin{itemize}
	\item Divide-and-Conquer Ansatz (in linke/rechte Hälfte teilen und beide mischen ($\mathcal{O}(n)$) 
	benötigt insgesamt $\mathcal{O}(n \log n)$
	\item Fortune-Sweep (Sweepline-Algorithmus) benötigt $\mathcal{O}(n \log n)$.
\end{itemize}

\subsection*{Delaunay-Triangulierungen}
\begin{itemize}
	\item Dual zum entsprechenden Voronoi-Diagramm
\end{itemize}

\section*{Suche in ebenen Unterteilungen}
\subsection*{Post Office Problem}
Datenstruktur? Anfragezeit etc.?

\section*{Sweepline-Verfahren}
\begin{itemize}
	\item Wir verwenden $SLS$ (Sweepline Status) Datenstruktur. Diese gibt Aufschluss über den
	"`Zustand des Algorithmus"' an der aktuellen Position der Sweepline. Diese ist in der Regel
	eine Wörterbuch-Datenstruktur, z.B. ein balancierter Suchbaum (z.B. AVL-Baum) 
	\item Es werden Event Points definiert und behandelt.
	\item Es gibt eine weitere Datenstruktur $EPS$ (Event-Point-Schedule), mit 
	der entschieden wird, was der nächste Event Point ist. Dies ist in der Regel eine 
	Priority-Queue, d.h. Heap o.ä.. Diese wird unter Umständen auch laufend aktualisiert.
\end{itemize}
\subsection*{Schnitte von $n$ Strecken in der Ebene}
Die Menge aller Schnittpunkte von $n$ Strecken kann mit dem Sweepline-Verfahren in $\mathcal{O}(n \log n + k)$ bestimmt werden, wobei $k$ die Anzahl der Schnittpunkte. (Denn es gibt $2n + k$ Ereignispunkte
und bei jedem Ereignis werden konstant viele Operationen auf den Datenstrukturen $SLS$ bzw. $EPS$ 
(jeweils logarithmische Zeit) ausgeführt.)
\subsection*{Triangulierung des Inneren von einfachen Polygonen}
Geht mit Sweepline in $\mathcal{O}(n \log n)$.
\subsection*{Fortune Sweep (Voronoi-Diagramm)}
\begin{itemize}
	\item $EPS$-Struktur: Wörterbuch
	\item Event Points: Punkt der Eingabemenge wird überstrichen (Site Event)  
	und Circle-Event (Schnitt von Spikes) 
	\item Site Event: Die getroffene Welle wird gespalten und neue Welle/Parabel wird 
	dazwischen eingefügt (in $SLS$ in $\mathcal{O}(\log n)$).
	Außerdem wird/werden eventuell? ein neues/neue Circle-Event(s) in $EPS$ eingefügt (rechtester Punkt des Kreises
	mit Mittelpunkt bei eventuellem Schnittpunkt der Spikes)
	\item Circle-Event: Zugeordnete Welle wird wieder entfernt.
	\item Laufzeit ist $\mathcal{O}(n \log n)$ da jeder überstrichene Punkt die Anzahl der Wellen
	maximal um 2 erhöht (Anzahl der Wellen $\mathcal{O}(n)$). Jedes mal finden außerdem nur konstant viele
	Operationen an den Datenstrukturen mit jeweils logarithmischer Zeit statt.
\end{itemize}

\subsection*{Berechnung der Vereinigung von Rechteckflächen}
\begin{itemize}
	\item $SLS$: Segmentbaum, welcher Projektionen der geschnittenen Rechtecke in x-Richtung enthält.
	\item Event Points: x-Koordinate der Ecken der Rechtecke
	\item $EPS$: Heap der diese enthält.
\end{itemize}
\section*{Geometrische Datenstrukturen}
\subsection*{k-d-Baum}
Beantwortet orthogonale Bereichsanfragen.
\begin{itemize}
	\item Anfragezeit: $\mathcal{O}(n^{1-1/d} + k)$
	\item Platzbedarf?
	\item Vorverarbeitungszeit?
\end{itemize}
\subsection*{Range-Tree / Bereichsbaum}
Beantwortet orthogonale Bereichsanfragen.
\begin{itemize}
	\item Anfragezeit: $\mathcal{O}(\log^d n + k)$
	\item Platzbedarf: $\mathcal{O}(n \log^{d-1} n)$
	\item Vorverarbeitungszeit: $\mathcal{O}(n \log^{d-1} n)$
\end{itemize}
Dynamisierung ist z.B. mittels Verwendung eines AVL-Baumes möglich.
\subsection*{Segmentbaum}
Ermittelt für Anfragepunkt in welchen Intervallen er liegt.
Dabei gibt das Universum $U$ mit $|U| = N$ vor, welche Intervallendpunkte "`erlaubt"' sind.
Die Menge $S$ mit $|S| = n$ enthält alle Intervalle, welche in den entsprechenden Knotenlisten
 gespeichert sind.
\begin{itemize}
	\item Jedes Intervall gehört zu $\leq 2h$ Knotenlisten
	\item Vorverarbeitungszeit: $\mathcal{O}(N + n \log N)$
	\item Platzbedarf: $\mathcal{O}(N + n \log N)$
	\item Anfragezeit: $\mathcal{O}(\log N + k)$
	\item Einfügen von Intervallen: $\mathcal{O}(g(n) \log N)$, wobei $g(n)$ die Zeit ist, die benötigt wird, um Intervalle in die Knotenlisten einzufügen.
\end{itemize}

\section*{Konvexe Hülle - Algorithmen}
\subsection*{Algorithmen für $\mathcal{R}^3$}
\subsubsection*{Gift Wrapping Algorithmus}
\begin{itemize}
	\item Laufzeit: $\mathcal{O}(n \log n)$, in höheren Dimensionen aber schlimmer..
\end{itemize}
\subsubsection*{Randomisierter Inkrementeller Algorithmus}
\begin{itemize}
	\item Erwartete Laufzeit: $\mathcal{O}(n \log n)$
	\item Worst Case Laufzeit: $\mathcal{O}(n^2)$
\end{itemize}

\section*{Dualisierung}
\subsection*{Konvexe Hülle mit Dualisierung lösen}
\begin{itemize}
	\item $H^+$ und $H^-$ (Obere und untere Halbebenen)
	\item Verwende Dualisierung von Eckpunkt 
	\item Im letzten Skript 2009 nochmal im Detail nachlesen!..
\end{itemize}

\subsection*{Delaunay Triangulierung}
Die Delaunay Triangulierung im $\mathcal{R}^2$ kann durch Berechnung der konvexen Hülle
im $\mathcal{R}^3$ berechnet werden.

Hierzu werden Punkt aus dem $\mathcal{R}^2$ auf den Einheitsparaboloid $P$ im $\mathcal{R}^3$ 
projiziert. Die Idee ist, dass wenn Punkte im $\mathcal{R}^2$ auf einem Kreis liegen und diese
wie beschrieben projiziert wurden dann im $\mathcal{R}^3$ in einer gemeinsamen Ebene liegen.
Jedes Dreieck der Delaunay Triangulierung entspricht einer Facette der konvexen Hülle im $\mathcal{R}^3$.

Algorithmus:

\begin{itemize}
	\item Bilde Projektion
	\item Konstruiere die konvexe Hülle im $\mathcal{R}^3$
	\item Projiziere die Kanten der konvexen Hülle  (ohne die Facetten welche die konvexe Hülle
	nach oben begrenzen) zurück nach $\mathcal{R}^2$
\end{itemize}

Genaue Erklärung nochmal nachlesen und genau verstehen!..

\section*{Zusammenfassung der Übungszettel}
\begin{itemize}
	\item Zettel 1
	\begin{itemize}
		\item Geradendarstellung (Hessesche Normalform, Nullstellenmenge, affine Hülle)
		\item Algorithmus mit BHD-Baum: Liegt p in P? Schnitt mit Geraden g? (log n)
	\end{itemize}
	\item Zettel 2
	\begin{itemize}
		\item Rotating Calipers (Durchmesser eines Polygons mithilfer der antipodalen Paare bestimmen)
		\item Konvexe Hülle in $\mathcal{R}^2$ mit Divide and Conquer (d.h. mit Rotating Calipers und 
		"`Brückenfindung"'
	\end{itemize}
	\item Zettel 3
	\begin{itemize}
		\item Graham-Scan Implementation
		\item Inkrementelle Konstruktion der konvexen Hülle in $\mathcal{R}^2$ (Nochmal anschauen!!)
		\item Untere Schranke für das Berechnen der konvexen Hülle durch Reduzieren des Sortierungsproblems auf das Berechnen der konvexen Hülle (Momentenkurve bzw. Parabel!)
	\end{itemize}
	\item Zettel 4
	\begin{itemize}
		\item Bewegungsplanung in der Ebene (Algorithmus für Roboter entsprechend eines Voronoi-Diagramms angeben)
		\item Geometrische Graphen (Beweis von Eulers Formel und Beweis, dass jede Triangulierung
		einer Menge von $n$ Punkten, von denen $r$ extrem sind, genau $2(n-1) - r $ Dreiecke enthält)
		\item Beschreibe einen Algorithmus welcher aus einem Voronoi-Diagramm $VD(S)$ in linearer
		Zeit den Rand der konvexen Hülle $CH(S)$ berechnet. (Jede Voronoi-Region als zyklische Folge
		der Kanten gegeben)
		\item Kann man das Voronoi-Diagramm von $S$ schneller als in $\Omega(n \log n)$ berechnen?
	\end{itemize}
	\item Zettel 5	
	\begin{itemize}
		\item Suchen in ebenen Unterteilungen 
		
		Hier sollte man eine "`einfache"' Datenstruktur mit Anfragezeit $\mathcal{O}(\log n)$ 
		beschreiben, welche einen Punkt in einer ebenen Unterteilung findet. Vorverarbeitungszeit und 
		Speicherplatz soll auch bestimmt werden. (Hinweis: Ziehen sie durch jeden Knoten der ebenen
		Unterteilung eine vertikale Gerade)
		
		\item Voronoi-Diagramme mit der $L_1$, also Manhattan-Metrik beschreiben und skizzieren
		\item Datenstruktur zur Punktlokalisierung in ebenen Unterteilungen (LDS)
		
		Diese sollte auch für unbeschränkte Facetten angepasst werden. (Großes Dreieck herumlegen)
		Vorverarbeitungszeit, Speicherbedarf und Anfragezeit sollte in Abhängigkeit der Anzahl
		der Knoten angegeben werden.
	\end{itemize}
	\item Zettel 6
	\begin{itemize}
		\item Voronoi-Diagramme von Strecken
		
		Wie sehen die Kurven aus?, Aus wie vielen Ecken, Kanten und Zellen kann $VD(S)$ höchstens
		bestehen? Zeigen Sie dass die Voronoi-Regionen zusammenhängend sind.
		
		\item Fortune-Sweep illustrieren und beschreiben (z.B. Was passiert an Eventpunkten etc.)
		
		\item Durchschnitt einfacher Polygone
		
		Geben Sie einen Sweepline-Algorithmus an und anaysieren Sie. (Es gibt glaube ich einen Algo
		in dem E-Book zur Veranstaltung..)
	\end{itemize}
	\item Zettel 7
	\begin{itemize}
		\item Anfragezeit für k-d-Bäume..
	\end{itemize}
	\item Zettel 8
	\begin{itemize}
		\item Zeigen Sie dass ein 2-dimensionaler Bereichsbaum in $\mathcal{O}(n \log n)$ aufgebaut
		werden kann.
		\item Beschreiben Sie das Einfügen/Streichen von Intervallen bei dynamischen Segmentbäumen
		\item Punkt-Rechteck-Anfragen
		Geben Sie eine Datenstruktur zum effizienten Auffinden einer Menge von Rechtecken zu einem
		Anfragepunkt $p$ an. (Segmentbaum mit sekundären Segmentbäumen..)
	\end{itemize}
	\item Zettel 9
	\begin{itemize}
		\item Beweisen Sie, dass es in 3d höchstens 5 platonische Körper gibt
		\item Die Ecken und $d-1$-dimensionalen Facetten vom Einheitswürfel bzw. 
		Einheitssimplex angeben und Hypercube illustrieren.
		\item Einfach zu beschreibender Brute-Force-Algorithmus zur Berechnung der konvexen
		Hülle in 3d. (Nehme alle 3-Punkte-Kombinationen und teste, ob alle anderen Punkte im jeweiligen
		Halbraum liegen (Binomialkoeffizient))
	\end{itemize}
	\item Zettel 10
	\begin{itemize}
		\item Dualisierung
		
		Möglichst effizienten Algorithmus der bestimmt, wie viele Punkte aus einer Menge $S$ 
		maximal auf einer gemeinsamen Geraden liegen.
		
		\item Für ein beliebiges $n$ eine Punktmenge im $\mathcal{R}^4$ angeben, deren konvexe
		Hülle Größe $\Omega(n^2)$ hat. 
		
		\item Eingaben/Einfügereihenfolgen für inkrementelle Konstruktion der konvexen Hülle bzw. 
		der Trapezzerlegung eines Arrangements von Strecken angeben, so dass die Laufzeit
		$\Omega(n^2)$ beträgt.
	\end{itemize}
	\item Zettel 11
	\item Zettel 12
\end{itemize}
\end{document}